\section{Definitions}




\begin{definition}[Siumatgwoon]\label{def:siumatgwoon}
    A Siumatgwoon, or a Metaphysic, is a set $S$, paired with the binary relations \textit{composition} $* : S \times S \rightarrow S$ and \textit{constitution} $|:S\times S \rightarrow \{\text{True}, \text{False}\}$, such that the following axioms hold. 

    \begin{axiom}[Reflexivity]\label{ax:reflex}
    For all $a \in S$, $a|a$.
    \end{axiom}

    \begin{axiom}[Totality]\label{ax:total}
    For all $a,b\in S$, exactly one of the following holds: $a|b$ or not $a\not|b$
    \end{axiom}

    \begin{axiom}[Transitivity]\label{ax:trans}
    For all $a,b,c\in S$, $a|b$ and $b|c$ implies $a|c$.
    \end{axiom}

    \begin{axiom}[Composition Constitution]\label{ax:comp-const}
    If $a*b=c$ for some $a,b,c\in S$, then $a|c$ and $b|c$.
    \end{axiom}
\end{definition}

$a * b$ should be read is "$a$ composing with $b$", and $a | b$ should be read as "$a$ constitutes $b$". You can also read $*$ as "and", "along with", "mixed with", "reacting with", "making love with", etc. You can also read $|$ as "is a part of", "composes", "is a constituent of", "is a component of", "is a sub-object of", etc. Because we are exploring, we will enable maximal leeway in the interpretation of $*$ and $|$, and this we will see, can yield some happy and interesting structures. 

Some further clarifications: 

\begin{itemize}
\item $a\not|b$ iff $a|b = \text{false}$
\end{itemize}

The above 4 axioms are the core axioms of a Siumatgwoon. There are quite a lot of ways in which you can think about what they are. You can think of them as sure foundations you can rest upon and then launch to explore the space, or you can think of them as a set of definitions that define the metaphysics of an object, in this case a mathematical object. But all mathematical objects are metaphysical objects (though we don't know if all metaphysical objects are mathematical objects - I'd wager not - the first go-to reason one might want to appeal to is Godel's Incompleteness Theorem - I think that's the right direction but Godel won't give us the direct proof to support this intuition - it is likely going to be something beyond mathematical language. Indeed, I think there are metaphysical objects that can be described by natural language, but not mathematical language. But I digress). We can further define other types of Siumatgwoons with further axioms. A taxonomy would then emerge, defined by various properties captured by the additional axioms.

But to capture the logic inside the Sinoglyphs, which is where we are starting - and we are starting there because we believe there's an interesting logic that evolved in there - just like interesting creatures and biosphere usually evolve in isolated and unique environments. I'To specify certain more restrictive structures, we will now introduce some more axioms, that makes a siumatgwoon simple. They are "simple" because they conform to our mereological intuitions, of both the Sinoglyphs and of the world.

\begin{definition}[Simple Siumatgwoon]\label{def:simple}
A Simple Siumatgwoon $S$ is a siumatgwoon where:
\begin{axiom}[Antisymmetry]\label{ax:antisymmetry}
    For all $a,b\in S$, if $a|b$ and $b|a$ then $a=b$.
\end{axiom}
\begin{axiom}[Finite composition]\label{ax:finite_composition} 
    Every object is finitely composed. For any $x\in S$, $x = y_1 * y_2 * y_3 \cdots y_n$ for some finite number $n$.
    In other words, there is no object that is composed of infinitely many objects.
\end{axiom}
    
\begin{axiom}[Finite Constitution]\label{ax:finite_constitution} 
For any $x \in S$, there are only finitely many objects $y_1, y_2, \ldots, y_n \in S$ such that $y_1, y_2, \ldots, y_n | x$.
\end{axiom}

\begin{axiom}[Unique Decomposition]\label{ax:unique-decomposition} 
    For any $x\in S$, the longest decomposition into objects of $S$ is unique. We can speak of the longest decomposition of an object, because we have axiom \ref{ax:finite_composition}, which says that every object is finitely composed.
\end{axiom}
\end{definition}

Axiom \ref{ax:antisymmetry} antisymmetry says no two distinct objects constitute each other. It rules out circular identity via mutual constitution. So you can't have chains like atoms|humans|universe|atoms - no embedded universes in atoms like in \ruby{手塚 治虫}{おさむ てずか}'s 《\ruby{火の鳥}{ひのとり}》 

Axiom \ref{ax:finite_composition} finite composition is a cardinality constraint on the composition operator $*$. Antisymmetry is a constraint on the constitution relation $|$. There's no logical connection between a limit on the number of components in a composition and antisymmetry.

Note that antisymmetry says nothing about how many objects compose another—only about how $|$ relates distinct objects. You could easily construct a model where antisymmetry holds, but some elements are infinitely composed (e.g., think of $x = a_1 * a_2 * a_3 * \cdots$ in a naïvely defined algebra).

One should also note that Axiom \ref{ax:antisymmetry} antisymmetry also rules out the possibility of a simple siumatgwoon modeling objects that obey sentences like "man is a part of nature, and nature is a part of man" or "the Tao is part of everything, and everything is part of the Tao" - as clearly the philosophies or metaphysics of those two utterances do not admit that just because $\text{man} | \text{nature}$ and $\text{nature} | \text{man}$, then $\text{man} = \text{nature}$, or that $\text{Tao} | \text{everything}$ and $\text{everything} | \text{Tao}$, then $\text{Tao} = \text{everything}$. 


Axiom \ref{ax:finite_constitution}, finite constitution, is a cardinality constraint on the constitution relation $|$. The point of both \ref{ax:finite_composition} and \ref{ax:finite_constitution} is to ground us in the world of finite compositions and finite constitutions. 

It says you cannot make a superobject by composing infinitely many objects - composition is finite. Obviously, one can eventually relax this and figure out how things will look. Those sensitive to theological arguements might already detect that finite composition precludes the existence of 㕦 in a Siumatgwoon so defined. If God is infiniite in the sense it has infinitely many attributes, then it is not possible to model God in a simple siumatgwoon. However, it seems perfectly intuitive to argue that all things have infinitely many attributes, so nothing can actually be modeled by a simple siumatgwoon. Perhaps a fruitful way to then model these objects of infinite attributes, is not to relax the axiom, but to consider a sequence of simple siumatgwoons $\text{兆}_0,\text{兆}_1, \text{兆}_2, \text{兆}_3, \ldots$, that somehow converge, where the limiting object is indeed an object of infinite attributes. Obviously, to define some kind of limiting mechanism or concept of converge, you'd need to have some concept of open sets, or stronger still, a metric. Let's not get into that. 

% Suppose we want to model the Thames, $\text{河}_{\text{Thames}}$, and in $\text{兆}_0$, we have 
% $$\text{兆}_0 = \{\text{河}, \text{河}_{\text{Thames}}\}$$
% $$\text{兆}_1 = \{\text{河}, \text{河}_{\text{Thames}}\}$$
% In $\text{兆}_0$

It will be very interesting to see what kind of objects will be yielded if we relax these two axioms - to infinite compositions and infinite constitutions. How will they look like?

Well, if we relax \ref{ax:finite_constitution}, one can have a siumatgwoon $S$ where everything can be infinitely decomposed. So there are no atoms - everything can be cut smaller and smaller and smaller - without end.

This is also a cardinality restriction—this time on the domain of constitutors of any object. Antisymmetry only constrains symmetric pairs in $|$, not how many $y$'s can point to $x$.

A model could satisfy antisymmetry yet have an object $x$ with infinitely many distinct constituents $y_i$ such that $y_i|x$, without any pair $y_i, y_j$ satisfying $y_i|y_j \wedge y_j|y_i$.

The uniqueness decomposition axiom, axiom \ref{ax:unique-decomposition}, is a uniqueness constraint on the decomposition of an object into its constituents. It says that the longest decomposition of an object is unique. It says that nothing can made of up of two sets of objects. If you have two different sets of components, then it's not possible to make the same thing. It says: among all decompositions of $x$, there is a unique one of maximal length (i.e. atomic resolution). This requires some structural property of $*$, perhaps associativity or atomic irreducibility. 

Antisymmetry doesn't regulate the structure of compositions—just the equality condition under mutual constitution. Even with antisymmetry, multiple maximal-length decompositions could exist, e.g., due to commutativity or ambiguity in decomposition paths, unless you restrict the structure of $*$.



These are some really weird objects indeed. But they are definitely not anything like the Sinoglyphs - at least not right now. 




Does the size of the infinity make things even more interesting? Careful now, we are trying to do what Leibniz didn't do, not to roleplay Cantor. 

Before we discuss some firther definitions, of Elementals, Generators, Atomics, let us remark some pattens we see in the Sinoglyphs. 

The first thing to remark is that Westerners have translated the 邊旁 into "radicals" - as if they're solutions to a quadratic equation, which means then they thought the whole sinoglyph itself is like an equation to be solved. 

If we take these radicals, that compose a sinoglyph, we'd say that they are the elementals of the sinoglyph. 

The way I came up with the set of definitions for atoics, elementals, generators - was to model the Sinoglyph writing system. Oftentimes, in the Sinoglyph system, as it appears to us, all three sets are the same. 

But if we consider a possible extension to the Chinese Character writing system, and simply consider the idealised and fully evolutionarily searched space of possible evolutionary trajectories of the Chinese character writing system, we can arrive at an idealised symbol combination system - that's the Sinoglyph. That Sinoglyph symbol manipulation system, has a structure. And that structure is its Siumatgwoon. And we are now trying to model that structure with some axioms. 

Atomics are basically 獨體字. They stand alone. They cannot be decomposed, and nothing constitutes them besides themselves.

Elementals are basically any sinolgyph that cannot be fully decomposed. Something else might constitute it, but its constitutes can be composed to make the whole the elemental. There's something more in the elemental than just whatever constituting. An elemental might have many many constituents, but the elemental is somehow larger than the composition of its parts. This can be called emergence. 

But not all elementals exhibit emergence. Because all atomics are elementals. Since atomics are constituted by themselves, and the atomic needs nothing to be composed with to be itself, atomics are elementals with no emergence. 


The elementals largely correspond to the kind of compound sinoglyphs 合體字 that are called phonosemanophores - 形聲字。 

Now let's think about how a 形聲字 works. Let's take the classic example of 江河, specifically 河. The Chinese character 河 is a 形聲字. 
$$\text{河}$$
It is composed of 氵(water) and 可 (ho). Ho is just a name. You might as well right it like this $$\text{氵}_{\text{可}}$$. But the sound is immaterial, it's just an index. It might as well be $a,b,c$

$$\text{氵}_{a},\text{氵}_b,\text{氵}_c$$
But the important thing is that picks something out. It picks out from the set of all objects that are constituted by water the thing that is "river". You might as very well write it as $$\text{氵}_{river}$$. 

The Chinese way of writing "river" is not different from writing it as "氵river". It says that "river" belongs to the class of things that are constituted of "氵"(water); it says "river" is a "氵" thing; it says "river" belongs to the set, the class, the category of "氵" things; but most importantly, it says: I don't know how "river" is composed, but I do know it is constituted of "氵". So there's something beyond water that makes a river - but I don't know what all the stuff that composes a river is, or hanc marginis exiguitas non caperet.

What this immediately suggest is a 系 system. A 金木水火土系 system. A system where elementals can be long to 金系, 木系, 水系, 火系 土系 system is a Siumatgwoon, where each 系 contains (but is not limited to) elementals that are constituted of the well, the "Head" of the 系. So the head, of a 土系 is 土. The head of a 金系 is 金. The head of a 木系 is 木. The head of a 水系 is 水. The head of a 火系 is 火. 

We want to capture this - the set of all objects exhibiting emergence from a certain "head". Let us call this set of head x the x-系. 



\begin{definition}[系 (Hai) Elements]\label{def:hai-elements}
    The set x-系, written as $\langle x \rangle$ is defined to be 
    
    $$\langle x \rangle := \{ s\in S: x|s \text{ but } \not \exists y_1, y_2,\ldots, y_n \in S \text{ such that } s=x*y_1*y_2*\cdots*y_n\}$$. 
    \end{definition}

    x-系 elements roughly correspond to the 形聲字 in Chinese characters - if you do not consider the phonetic component of a character to be full elements inside the Siumatgwun but mere indices. In other words, one does not view 江、河、湖、海 as 水工、水可、水胡、水每 but as $\text{水}_\text{工}、\text{水}_\text{可}、\text{水}_\text{胡}、\text{水}_\text{每}$. In this sense, one can see there is no element $y \in S$ such that $\text{水}y=\text{水}_\text{工}=\text{江}$.

        
 So what's in an x-系? Say, 木系, the set $\langle \text{木} \rangle$?

Well, it contains $\text{木}$, because $\text{木}$ is cannot be further decomposed. But we also have the other trees, 松,柏,柳,桃,梅, so we can say 

$$\text{木}_{\text{公}},\text{木}_{\text{白}},\text{木}_{\text{卯}},\text{木}_{\text{兆}},\text{木}_{\text{每}},  \text{木} \in \langle \text{木} \rangle$$



But note that $\text{林}, \text{森},  \text{焚} \not\in \langle \text{木} \rangle$

Does the head of a 系 have to be an atomic element? No, not at all! In fact, the head of a 系 could itself be an elemental like 河, or a proper compound like 雲. Consider the siumatgwoon $S = \{\text{雨}, \text{云}, \text{雲}, \text{霒}, \text{霕}, \text{霴}, \text{靉}\}$. In this set clearly the atoms are $\{\text{雨}, \text{云}\}$, but the elementals are $\{\text{雨}, \text{云}, \text{霒}, \text{霕}, \text{霴}, \text{靉}\}$. So $\{\text{霒}, \text{霕}, \text{霴}, \text{靉}\}$ are all in $\langle \text{雨} \rangle$ and in $\langle \text{云} \rangle$ and also in $\langle \text{雲} \rangle$.

But there's no reason why elementals, 形聲字, themselves, cannot be the head of a 系. Why shouldn't there be say an object $x=\text{河}_{a}$ or $river_{a}$ that we know is a kind of river, but we don't know what all the other constituents are to make it more than just a "typical"river? I mean, it's pretty obvious that The Thames, La Seine, The Hudson, The Jyugong, are all rivers, i.e.: 
$$\text{河} | Thames, Seine, Hudson, \text{珠江}$$

Which is to say 
$$\text{河} |  \text{河}_{Thames},\text{河}_{Seine},\text{河}_{Hudson},\text{河}_{\text{珠江}} $$
but we can't write down all the other constituents that make them more than just a river, and not how they differ from each other either. So, 
i.e.: 
$$\text{河}_{Thames},\text{河}_{Seine},\text{河}_{Hudson},\text{河}_{\text{珠江}} \in \langle \text{河} \rangle$$



Let us now get some proper definitions.


\begin{definition}[Atomics]\label{def:atomics}
    An element $a \in S$ is atomic if for all $x\in S$, if $x|a$ implies $x=a$, i.e. $a$ is atomic iff only anything that constitutes $a$ is $a$ itself.
\end{definition}

\begin{definition}[Elementals]\label{def:elementals}
    An element $e \in S$ is elemental if and only if for all $x_1, x_2, x_3, \ldots, x_n \in S$ where $x_1, x_2, x_3, \ldots, x_n | e$, we have cannot find a finite sequence drawn from $x_1, x_2, x_3,\ldots, x_n $ such that it composes $e$. In other words, an elemental has no non-trivial decomposition other than $e = e$. 
\end{definition}


Note that Siumatgwoons do not swear alliegance to commutativity or associativity. $a*b$ might not equal to $b*a$ and $a*(b*c)$ might not equal to $(a*b)*c$. Both sucrose and fructose have the same molecular formula of $\text{C}_6\text{H}_{12}\text{O}_6$, and so $\text{C}, \text{H}, \text{O} | sucrose, fructose$, but sucrose and fructose are not the same object. Listing the constituents of an object also does not specify how those individual constituents are themselves composed to form components that form the object, which means we don't actually know how many times $C, H, O$ appear in sucrose or fructose. The point of the definition of the elemental, is to say, no matter how you arrange the constituents, no matter how many times each of the constituents appear in a composition, you cannot compose the object from those constituents alone. Again, emergence is at play here.



We quickly define the sets of atomics $A$ and the set of elementals $E$.

\begin{definition}[Atomic Set]\label{def:atomic-set}
    The set $S_A \subseteq S$ is called the atomic set of $S$. It is the set of all atomic elements inside $S$.
\end{definition}

\begin{definition}[Elemental Set]\label{def:elemental-set}
    The set $S_E \subseteq S$ is called the elemental set of $S$. It is the set of all elemental elements inside $S$.
\end{definition}


\begin{lemma}[Atomics are Elementals in Simple Siumatgwoons]\label{def:atomics-are-elementals-in-simple-siumatgwoons}
    All atomics are elementals in simple siumatgwoons. For all simple siumatgwoons $S$, $A \subseteq E$.
\end{lemma}
\begin{proof}
    Let $a \in A$. $a$ clearly constitutes itself, so $a=a$. If $a$ is not an elemental, then there exists a decomposition $a=b*c$ for some $b,c \in S$. By atomicity, $b=a$ or $c=a$. But then we can apply the same argument again, so $a=a*a$. And we can go ad infinitum, so $a=a*a*\cdots$. But this contradicts \ref{ax:finite_composition}, which states every object in a simple siumatgwoon are composed by a finite number of objects. Therefore, $a$ cannot have any decomposition other than $a$. Therefore, $a$ is an elemental.
\end{proof}


\begin{lemma}[Atomics]\label{def:atomics}
    The decomposition of an atomic in a simple Siumatgwoon is unique. i.e. $a=a$ is the only decomposition of an atomic.
\end{lemma}
\begin{proof}
    As proven above.
\end{proof}



\begin{definition}[Generators]\label{def:generators}
    A set $G \subseteq S$ is a generator of $S$ if and only if every object $x \in S$ can be expressed as a composition (trivial or compound) of objects from $G$.
\end{definition}

\begin{theorem}[Generator sets exist in simple siumatgwoons]\label{thm:generator-sets-exist-in-simple-siumatgwoons}
    In a simple siumatgwoon, there exists a non-empty subset $G_{S} \subseteq S$ such that every element $x \in S$ is a product of a finite sequence of elements from $G_S$, i.e., $x = g_1 * g_2 * \cdots * g_n$, where $g_1, \dots, g_n \in G_S$.
\end{theorem}

\begin{proof}
    This is clearly true, because any object $x\in S$ is always a finite composition of itself. So we can just let $G_S = S$. In other words, the set of all objects in $S$ is a generator set of $S$.
\end{proof}

\begin{lemma}[Elementals are in every generator set]\label{lem:elementals-are-in-every-generator-set}
    In a simple siumatgwoon, $E\subseteq G$ for any generator set $G$.
\end{lemma}
\begin{proof}
    Since any generator set $G$ must generate all objects in $S$, the generator set $G$ must generate all elementals in $E$. But elementals have non-trivial decompositions, so they are generated only by themselves. Therefore, to generate the elementals, the generator set $G$ must contain all elementals. Therefore $E\subseteq G$. This also means no generator set can be smaller than $E$.
\end{proof}

\begin{lemma}[All objects are either elementals or compounds]\label{lem:elementals-are-not-compounds}
    For any $x\in S$, $x$ is either an elemental or a compound.
\end{lemma}
\begin{proof}
    If $x$ is an elemental, then it is an elemental. If $x$ is not an elemental, then it has a non-trivial decomposition. Then it is a compound.
\end{proof}



\begin{lemma}[Compounds are composed by elementals]\label{lem:compounds-are-composed-by-elementals}
    If an object $x\in S$ is a compound, then it is composed by elementals.
\end{lemma}
\begin{proof}

    If $x$ is a compound that cannot be decomposed into only elementals, then in any decomposition of $x$ you can always find a compound $c_1$ in the decomposition. Because $c_1$ is a compound, it can also be decomposed, and in any decomposition of $c_1$ you must always be able to find a compound $c_2$ in the decomposition, because if otherwise, $c_1$ will be composed entirely of elementals, and so will $x$. But the argument applies for $c_2$ as well, and you will be able to find a compound $c_3$. This can be done indefinitely and you will get a sequence of $c_1, c_2, c_3, \ldots$, and by their nature, we have  $\ldots, c_1|c_2|c_3|\cdots|x$, and they are all distinct. This means, for $x$, we can find a infinite set of $c_1, c_2, c_3, \ldots$, all constitutents of $x$. But this is in violation of axiom \ref{ax:finite_constitution}, the finiteness of constitution. Therefore, by contradiction, there can be no compound that is composed of only compounds - therefore, all compounds are composed of elementals.
\end{proof}


Note that in a non-simple siumatgwoon where compositions can go on definitely deep, you can always chop at some level, and whatever higher than the level chopped will yield a simple siumatgwoon - because that seems to be our universe. 

% \begin{theorem}[You must divide]\label{thm:the-emergence-set-of-a-set-is-a-subset-of-the-set}
%     If $c$ is a compound of two elementals, i.e. $c=e_1 * e_2$, and $x|c$ and $x\neq c$, then $x|e_1$ or $x|e_2$.
% \end{theorem}
% \begin{proof}
%    So $x|e_1*e_2$. Suppose $x$ constitutes neither $e_1$ nor $e_2$, i.e. $x\not|e_1$ and $x\not|e_2$.

   
% \end{proof}



Now we move to a stronger theorem.

\begin{theorem}[The elementals are a generator set in simple siumatgwoons]\label{thm:elementals-are-a-generator-set-in-simple-siumatgwoons} 
    In a simple siumatgwoon, the set of elementals $E$ is a generator set of $S$.
\end{theorem}

\begin{proof}
    We have proven $E\subseteq G$ for all generator sets $G$ in $S$. Now we need to prove that for any $x\in S$, $x = e_1 * e_2 * \cdots * e_n$ for some $e_1, \dots, e_n \in E$.

    From \ref{lem:compounds-are-composed-by-elementals}, we know that any compound $x$ is composed by elementals. So we can write $x = e_1 * e_2 * \cdots * e_n$ for some $e_1, \dots, e_n \in E$.

    But since every element in $S$ is either an elemental or a compound, and compounds are all composed by elementals, any element in $S$ is therefore a composition of elementals - in other words, generated by the elementals. Therefore $E$ is a generator set of $S$.
\end{proof}


And now we can prove the following theorem.
    
\begin{theorem}[Unique decomposition into elementals]\label{thm:elementals-unique-decomposition}
    Let $S$ be a Simple Siumatgwoon, and let $E$ be the set of elementals in $S$. Then every object in $S$ has a unique decomposition into elementals (up to ordering if composition is commutative; otherwise, strictly unique).
\end{theorem}
    
\begin{proof}
    By Theorem~\ref{thm:elementals-are-a-generator-set-in-simple-siumatgwoons}, every object in $S$ can be written as a finite composition of elementals. That is, for any $x \in S$, there exists a decomposition:
    
    $$x = e_1 * e_2 * \cdots * e_n$$
    
    with each $e_i \in E$.
    
    Since elementals are by definition indecomposable, no further refinement of this decomposition is possible. Hence, such a decomposition into elementals must be a longest decomposition of $x$.

    But by Axiom~\ref{ax:unique-decomposition}, every object in a Simple Siumatgwoon has a unique longest decomposition. Therefore, the decomposition of any object into elementals is unique.
\end{proof}
    
Now we let us return to x-系 sets, and begin with some simple properties

\begin{lemma}[An object is in its own emergence set]\label{lem:an-object-is-in-its-own-emergence-set}
    For any $x\in S$, $x \in \langle x \rangle$.
\end{lemma}
\begin{proof}
    By definition, since $x$, and there is no $y_1$, $y_2$, $\ldots$, $y_n$ such that $x=x*y_1*y_2*\cdots*y_n$. Therefore $x \in \langle x \rangle$.
\end{proof}



\begin{theorem}[Properties of emergence sets]\label{thm:properties-of-emergence-sets}
    For any $x,y\in S$, we have:
    \begin{itemize}
        \item $x \in \langle x \rangle$.
        \item $x*y \in \langle xy \rangle \subseteq \langle x \rangle , \langle y \rangle \subseteq \langle x \rangle \cup \langle y \rangle$
        \item if $x\in \langle y \rangle$, then $\langle x \rangle \subseteq \langle y \rangle$
        \item $x|y$ iff $\langle y \rangle \subseteq \langle x \rangle$
        \item $\langle x \rangle = \langle y \rangle$ iff $x =y$
        
                
    \end{itemize}
\end{theorem}



\begin{theorem}[The emergence set of a set is a subset of the set]\label{thm:the-emergence-set-of-a-set-is-a-subset-of-the-set}
    If $\langle x \rangle$ contains compounds if and only if $x$ is a compound. 
\end{theorem}
\begin{proof}
    We will prove the converse first, which is the obvious case. If $x$ is a compound, and given $x\in \langle x \rangle$, then $\langle x \rangle$ contains compounds. 

    Now suppose $c\in \langle x \rangle$ is an compound. And by definition $x|c$. But then since $c$ is a compound, $x$ 


\end{proof}


Then we can prove the following lemma. 
\begin{lemma}[The emergence of a constituted is emergence of the constituent]\label{lem:emergence-of-a-constituted-is-emergence-of-the-constituent}
    If $a|b$, then $ \langle b \rangle \subseteq \langle a \rangle$.
\end{lemma}
\begin{proof}
    If $a|b$, then for any $x\in \langle b \rangle$, because $a|b$ and $b|x$, we have $a|x$. And since $x$ is an elemental, it has no non-trivial decomposition. An object constituted by $a$ and has no non-trivial decomposition is an $a$-系 object, so $x \in \langle a \rangle$. Therefore, $\langle b \rangle \subseteq \langle a \rangle$
\end{proof}

Indeed, to illustrate this, consider the following:
$\langle \text{水}_\text{工} \rangle = \langle \text{江} \rangle \subseteq
\langle \text{水} \rangle$. 

Indeed,$\langle \text{靉} \rangle, \langle \text{霴} \rangle, \langle \text{霕} \rangle, \langle \text{霒} \rangle \subseteq \langle \text{雲} \rangle \subseteq \langle \text{雨} \rangle$

Or to take more interesting example, if we take $\text{江湖}$ is an object itself of itself, then the set $\langle \text{江湖} \rangle = \{\text{江湖}_{1}, \text{江湖}_{2}, \ldots\}$ obeys the following 

$\langle \text{江湖} \rangle \subseteq \langle \text{江} \rangle,  \langle \text{湖} \rangle \subseteq \langle \text{水} \rangle$

\begin{lemma}[Every elemental has an atomic constitutent]\label{lem:every-elemental-has-an-atomic-constituent}
    Every elemental $e$ has an atomic constituent $a$ such that $e|a$. In other words, for any $e \in E$, there exists an $a \in A$ such that $e \in \langle a \rangle$.
\end{lemma}
\begin{proof}
    Assume for contradiction that $e$ has no atomic constituents. 

    Then we know that there must be an $x$ such that $x$ is a constituent of $e$, and that $x\neq e$, because if for all $x$ such that $x|e$, $x=e$, then $e$ is atomic. 

    So let us assume there is an $x$ such that $x|e$ and $x\neq e$. $x$ is either an elemental or a compound. We know all compounds are composed of elementals. So, regardless of whether $x$ is an elemental or a compound, there is going to be an elemental $f_1$ such that $f_1|x$. Per assumption, this elemental $f_1$ must not be an atomic, so another elemental $f_2\neq f_1$ such that $f_2|f_1$. But the argument can be applied again, indefinitely. But this means, we obtain an infinite set of $f_1, f_2, \ldots$ such that $f_i|f_{i+1}$ for all $i$. But this contradicts \ref{ax:finite_constitution}, which states every object in a simple siumatgwoon are composed by a finite number of objects. Therefore, every elemental has an atomic constituent.
\end{proof}








Let us now extend definition of emergence and apply it to entire sets. 
    
\begin{definition}[X-系]\label{def:hais-of-sets}
        Let $X\subseteq S$. The $X$-系 is defined to be $$\langle X \rangle := \cup_{x\in X} \langle x \rangle$$. We call this set  $\langle X \rangle$ the emergence set of $X$, or the set of emergences from $X$.
\end{definition}

Clearly, for any $X\subseteq S$, $\langle X \rangle$ is a unique.

The X-系 set we will be particularly interested in, will be the A-系 set. 

\begin{axiom}[The emergence set of A]\label{ax:the emergence-set-of-A}
    The A-系 set, $\langle A \rangle$ defined as
     $$\langle A \rangle := \cup_{a\in A} \langle a \rangle$$ 
     is called the emergence set of $A$, or the set of emergences from $A$. 
\end{axiom}


There are some interesting properties of an emergence set: 

% \begin{theorem}[The emergence set of a set is a subset of the set]\label{thm:the-emergence-set-of-a-set-is-a-subset-of-the-set}
%     For any $X\subseteq S$, $\langle X \rangle \subseteq X$, we have:
%     \begin{itemize}
%         \item $\langle A
%     \end{itemize}
% \end{theorem}

% \begin{proof}
% SHIT
% \end{proof}

% \begin{theorem}[The emergence set of a set is a subset of the set]\label{thm:the-emergence-set-of-a-set-is-a-subset-of-the-set}



% \end{theorem}




This also implies, the following theorem. 
\begin{theorem}[Every 系 set is a subset of a 系 set whose head is atomic]\label{thm:every-hai-set-is-a-subset-of-a-hai-set-whose-head-is-atomic}
    For any $x\in S$, $\langle x \rangle \subseteq \langle a \rangle$ for some $a\in A$.
\end{theorem}
\begin{proof}
    Consider $\langle x \rangle$. $x$ is either an elemental or a compound. And every compound is made of elementals. So there exists an elemental $e$ such that $e|x$. Therefore $\langle x \rangle \subseteq \langle e \rangle$. 
    
    By \ref{lem:every-elemental-has-an-atomic-constituent}, we know that for any $e \in E$, there exists an $a \in A$ such that $a|e$, so we have $\langle e \rangle \subseteq \langle a \rangle$. Therefore, there is an atomic $a$ such that $\langle x \rangle \subseteq \langle a \rangle$
    
\end{proof}


\begin{theorem}[Indempotence of emergence]\label{thm:indempotence-of-emergence}
    For any $x\in S$, $\langle \langle x \rangle \rangle = \langle x \rangle$.
\end{theorem}
\begin{proof}
    To aid a sense of intuition, let us swap out $x$ with $\text{雲}$ for now. 

    Consider the definition, we have
    
    $$\langle \langle \text{雲} \rangle \rangle := \langle \{\text{雲}, \text{霒}, \text{霕}, \text{霴}, \text{靉}, \ldots, \text{雲}_{1}, \text{雲}_{2}, \ldots\} \rangle = \bigcup_{i} \langle \text{雲}_{i} \rangle$$ 

    Here perhaps we assumed that $\langle \text{雲} \rangle$ is countable... (we will one day deal with this.)

    Well, note that 

    $$\langle \langle \text{雲} \rangle \rangle := \langle \{\text{雲}, \text{霒}, \text{霕}, \text{霴}, \text{靉}, \ldots, \text{雲}_{1}, \text{雲}_{2}, \ldots\} \rangle = \langle \text{雲} \rangle \cup \bigcup_{\text{雲}_{i}\in 
    \langle \text{雲} \rangle - \{\text{雲}\}} 
     \langle \text{雲}_{i} \rangle$$ 

     And clearly, for every $\text{雲}_{i}$, we have $\text{雲}|\text{雲}_{i}$. Therefore, $\langle \text{雲}_{i} \rangle \subseteq \langle \text{雲} \rangle$. Therefore,$\cup_{\text{雲}_{i}\in 
    \langle \text{雲} \rangle - \{\text{雲}\}} \subseteq \langle \text{雲} \rangle$.
    
And so, we have          
     
    $$\langle \langle \text{雲} \rangle \rangle = 
     \langle \text{雲} \rangle \cup \bigcup_{\text{雲}_{i}\in 
     \langle \text{雲} \rangle - \{\text{雲}\}} 
      \langle \text{雲}_{i} \rangle
       = \langle \text{雲} \rangle \cup \langle \text{雲} \rangle
    =\langle \text{雲} \rangle$$

\end{proof}

% Another point: 
% \begin{theorem}[A set is in its emergence set]\label{thm:a-set-is-in-its-emergence-set}
%     For any $X\subseteq S$, $X\subseteq \langle X \rangle$.
% \end{theorem}
% \begin{proof}
%     $$\langle X \rangle = \cup_{x\in X} \langle x \rangle $$
% \end{proof}


Some properties of emergence sets: 
\begin{theorem}[The emergence set of a set is a subset of the set]\label{thm:the-emergence-set-of-a-set-is-a-subset-of-the-set}
    We have
    \begin{itemize}
        \item $\langle X \cup Y \rangle = \langle X \rangle \cup \langle Y \rangle$
        \item $\langle X \cap Y \rangle \subseteq \langle X \rangle \cap \langle Y \rangle$
        
        
    \end{itemize}
\end{theorem}
\begin{proof}
    SHIT
\end{proof}






Now we can prove something even more interesting. 
\begin{theorem}[The elementals is identical to A-系 in simple siumatgwoons]\label{thm:hais-of-sets-are-siumatgwoons}
    In a simple siumatgwoon, $\langle A \rangle = E$ for any $A\subseteq S$.
\end{theorem}
\begin{proof}
    We will first prove that $\langle A \rangle \subseteq E$, and then we will prove that $E \subseteq \langle A \rangle$.

    Let us prove that $\langle A \rangle \subseteq E$.  

    Well, clearly, for all $a$ where $a$ is an atomic, $a \in E$, because all atomics are elementals. Let us consider any $x\in \langle A \rangle$. By definition, $x$ is such that there is some atomic $a\in A$ such that $a|x$, but there is no way to decompose $x$ into a decomposition containing $a$. So $x$ is an elemental. 

    Now let us prove that $E \subseteq \langle A \rangle$.

Recall that for all elementals $e\in E$, there exists an atomic $a\in A$ such that $a|e$, by lemma \ref{lem:every-elemental-has-an-atomic-constituent}. Therefore, $e\in \langle a \rangle \subseteq \langle A \rangle$. Therefore, $E \subseteq \langle A \rangle$.
\end{proof}
