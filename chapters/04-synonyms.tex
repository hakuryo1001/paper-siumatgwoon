Consider the old siumatgwoon axioms, and a new relation "synonym", represented by $\sim$, read as "is synonymous with". It satisfies the following axioms.

\subsection{Synonym Axioms}

\begin{axiom}[Reflexivity]\label{ax:syn-reflex}
S1. (reflexivity) for all $x\in S$, $x\sim x$.
\end{axiom}

\begin{axiom}[Symmetry]\label{ax:syn-sym}
S2. (symmetry) For all $a,b\in S$, if $a\sim b$ then $b\sim a$.
\end{axiom}

\begin{axiom}[Transitivity]\label{ax:syn-trans}
S3. (transitivity) for all $a,b,c \in S$, if $a\sim b$, and $b \sim c$, then $a \sim c$.
\end{axiom}

\begin{axiom}[Compositional Congruence]\label{ax:syn-comp}
S4. (Compositional congruence) If $a \sim a'$, then (if they $ax$ or $xa$ exists):
\begin{itemize}
\item $ax \sim a'x$,
\item $xa \sim xa'$
\end{itemize}
\end{axiom}

\begin{axiom}[Composition Cancellation]\label{ax:syn-cancel}
S5. Composition cancellation
\begin{itemize}
\item If $a * b \sim a * c$, then $b \sim c$;
\item If $b*a\sim c*a$ then $b \sim c$;
\end{itemize}
\end{axiom}

\begin{axiom}[Divisor Compatibility]\label{ax:syn-div}
S6. Divisor compatibility: If $a \sim b$, then for all $x \in S$, $x|a$ iff $x|b$.
\end{axiom}

\begin{axiom}[Upwards Divisibility]\label{ax:syn-up}
S7: Upwards divisibility: If $a\sim b$, then for all if $a|x$ then $b|x$.
\end{axiom}

A Siumatgwoon is Synonym-Closed if in which if $a*b$ exists and $a\sim a'$ and $b\sim b'$ then $a'*b, a*b', a'*b'$ also exist. Most siumatgwoons, including the Sinoglyphs, are not synonym-closed. 

Because of S4, Synonym-Closed Siumatgwoons must have $a*b \sim a'*b \sim a*b'\sim a'*b'$.

\subsection{S8: Synonym Replacement and Existence Preservation}

Synonym replacement completion: if $a*b$ exists, and if $b\sim b'$, then $a*b'$ exists in $S$ as well.

Prior to this, we are rather ambiguous and coy as to whether $a$ and $b$ being synonymous means $b*x$ necessarily exists if $a*x$ exists. I feel we should admit closure for synonym substitutional compositions. We can consider other structure where this axiom is not admitted later.

S9. Multidecomposition if $ab \sim cd$ and $cd = e$ then $ab =e$. 

From S4 you'd be able to prove that $a \sim a'$ and $b\sim b'$ then $a * b \sim a' * b'$.

\section{The Quotient Siumatgwoon $S/\sim$}

\subsection{Starting Point: Synonymy $\sim$ as an Equivalence Relation}

Let $S$ be a Siumatgwoon, and let $\sim$ be a synonymy relation on $S$ satisfying:

\begin{itemize}
\item $\sim$ is an equivalence relation: reflexive, symmetric, transitive.
\item $\sim$ is \textbf{compatible with composition}:
    
    If $a\sim a'$, $b\sim b'$, then $a*b\sim a'*b'$.
    
\item $\sim$ is \textbf{compatible with divisibility}:
    
    If $a\sim b$, then $x|a \iff x|b$ for all $x\in S$.
\end{itemize}

These give us a solid foundation to define a quotient structure.

\subsection{The Set $S/\sim$}

Let:

$S/\sim=\{[a]:a\in S\}$

where $[a]=\{x\in S\mid x\sim a\}$ is the \textbf{equivalence class} of $a$.

\subsection{Defining the Operations on $S/\sim$}

\subsubsection{Multiplication}

We define:

$[a]*[b]:=[a*b]$, if $a*b$ exists.

This is well-defined \textbf{because of compositional congruence} (axiom S4).

It is also because that we required if $a*b$ exists, so must $a'*b'$.

That is: if $a \sim a'$, and $b \sim b'$, then:

$a*b\sim a'*b'\Rightarrow [a*b]=[a'*b']$

So the result is independent of the representative.

\subsubsection{Divisibility}

We define:

$[a]|[b] \iff a|b$

Again, this is \textbf{well-defined} thanks to the \textbf{divisibility equivalence} axiom (S6):

If $a\sim a', b \sim b'$, given we have the axiom that $a|b \iff a'|b'$ then:

$[a]|[b] \iff [a']|[b']$

Thus, $|$ descends to the quotient.

\subsection{Verifying Siumatgwoon Axioms in $S/\sim$}

Let's verify that the \textbf{quotient structure inherits} the original Siumatgwoon axioms:

\begin{itemize}
\item \textbf{Axiom 1 (Reflexivity of $|$)}:
    
    Since $a|a\Rightarrow[a]|[a]$
    
\item \textbf{Axiom 2 (Totality)}:
    
    For all $[a],[b]$, either $[a]|[b]$ or not.
    
    This follows since $|$ on $S$ has totality, and divisibility is preserved under $\sim$.
    
\item \textbf{Axiom 3 (Transitivity)}:
    
    Suppose $[a]|[b]$ and $[b]|[c]$. Recall the definition that $[a]|[b]$ then $a|b$. It's straightforward.
    
\item \textbf{Axiom 4 (Composition and divisibility)}:
    
    We want to prove that if $[a]*[b]=[c]$, then $[a],[b] | [c]$. By definition $[a]*[b]=[c]$ then $a*b\sim c$, and so $a|c$. By the definition of constitutiveness over synonym classes, $[a]|[c]$. The argument same goes for $[b]|[c]$.
\end{itemize}

Now come axioms 5 and axioms 6. Do they hold in $S/\sim$, even if they might not hold in $S$?

\begin{itemize}
\item Axiom 5 (Complete Constructibility and Generators): There exists a non-empty subset $G_{S} \subseteq S$ such that:
    \begin{itemize}
    \item (Generation) Every element $x \in S$ is a product of a finite sequence of elements from $G_S$, i.e., $x = g_1 * g_2 * \cdots * g_n$, where $g_1, \dots, g_n \in G_S$
    \end{itemize}

And a Simple Siumatgwoon is one where:

\begin{enumerate}
\item $E$ is a generating set, and 
\item that decompositions into elementals are unique.
\end{enumerate}

\item Axiom 6. (Finite constitution). For any $x \in S$, there are only finitely many objects $y_1, y_2, \ldots, y_n \in S$ such that $y_1, y_2, \ldots, y_n | x$.
\end{itemize}

Let's prove that axiom 5 holds first. And then axiom 6 holds. Then we prove that $S/\sim$ is a simple siumatgwoon.

Consider some element $[x] \in S/\sim$. Given $S$ is a siumatgwoon, we know that any $x$ has a finite decomposition $g_1 * g_2 * g_3 * \cdots g_n$. Given $x=g_1 * g_2 * g_3 * \cdots g_n$, we have $[x]=[g_1 * g_2 * g_3 * \cdots g_n]$, and by the definition of $[*]$ we have $[x]=[g_1 * g_2 * g_3 * \cdots g_n]=[g_1] * [g_2] * [g_3] * \cdots [g_n]$. Given this applies to every $x$ in $S$, it is true of every $[x]\in S/\sim$. 

Note that some of the $[g_i]$ might be equal to some other $[g_j]$ if $g_i \sim g_j$.

Now for axiom 6.

Consider $[x]\in S/\sim$. By axiom 6, there only finitely many $y_1, y_2, \ldots, y_n | x.$ Therefore there are only finitely many $[y_1], [y_2], \ldots, [y_n] | [x].$ Note again, that because there could be $y_i \sim y_j$, then the most reduced list of constituents for $[x]$ might be shorter than $n$.

\textbf{Axiom 5 (Generators)}:

If $G\subseteq S$ is a generating set, then $[G]=\{[g]\mid g\in G\}\subseteq S/\sim$ generates all of $S/\sim$

Hence, $S/\sim$ is a well-defined \textbf{quotient Siumatgwoon}.

\subsection{Interpretation of $S/\sim$}

\begin{itemize}
\item \textbf{Elements of $S/\sim$} are \textbf{semantic classes}: bundles of expressions that mean the same thing.
\item \textbf{Operations and structure} are all inherited from how the parts combine.
\item This is essentially a move from \textbf{syntax to semantics} — a form of canonical simplification.
\end{itemize}

\subsection{Identity and Atoms in $S/\sim$}

\begin{itemize}
\item The set of \textbf{atomic classes} would be:

$A/\sim=\{[a]\mid a\in A\}$

If synonymy is strong (e.g., if all synonyms of atoms are equal), then this is a \textbf{true set of atomic meanings}.

\item The quotient set \textbf{removes redundancy}: if two characters are written differently but function identically, they're collapsed.
\end{itemize}

\section{A Hierarchy of Synonym Relations}

Intuitively speaking, the $\sim$ of 言$\sim$文 that allows us write 信$\sim$伩 seems different from the $\sim$ in 女$\sim$巫, as it does not seem the case that 女 can be exchanged for 巫 in all Chinese characters. This suggests that there is a hierarchy, or rather, a family of synonymous relations, perhaps context dependent. 

Some synonyms are more closely clustered together than others. This suggests there is a hierarchy of synonyms.



But to interpret $~$ to only mean "is synonymous with", leaves us certain phenonomena in, or rather, certain ways to "play" Chinese characters, uncaptured. 

In Classical Chinese poetry and prose, it is appreciated if couplets are written in a way that characters in the same position belong to "some same class" or objects. 

Consider all these lines from 《增廣賢文》:


兩人一般心,無錢堪買金;一人一般心,有錢難買針。
馬行無力皆因瘦,人不風流只爲貧。

畫虎畫皮難畫骨,知人知面不知心。
錢財如糞土,仁義値千金。
流水下灘非有意,白雲出岫本無心。

路遙知馬力,日久見人心。
相見易得好,久住難爲人。
饒人不是癡漢,癡漢不會饒人。
有錢道眞語,無錢語不眞。
長江後浪推前浪,世上新人趕舊人。
來如風雨,去似微塵。
萬般皆是命,半點不由人。


寧可正而不足,不可邪而有餘。
寧可信其有,不可信其無。


To illustrate the point, consider this following sentence, taken from the Analects. 

萬方有罪,罪在朕躬。

"Whoever is sinful, the guilt is on my body."

This sentence clearly expresses a monarchist position of politics - one that is shared by both classical Confucianism and medieval Christian political theory. The sovereign, the 君主, is the one who bears the responsibility, and the sins of the 萬方.

Let us simplify and modify this sentence a bit: 

君主之國,萬方有罪,罪在君身。

"In a country where monarchs are sovereign, whoever is sinful, the guilt is on the monarch's body."

Now a we can apply a naive application of lax synonymy, say on 萬 ("ten thousand") and 兆 ("trillion"), drawing upon they can both be taken to mean "uncountably many" on a metaphoric level, so $\text{萬} \sim \text{兆}$, so we get 
$$
\text{君主之國,\underline{萬}方有罪,罪在君身}
\sim
\text{君主之國,\underline{兆}方有罪,罪在君身}
$$

We don't have the full algebraic machinery to justify this yet, but it is reasonable. 

What I really want to point out, is that in a sense, 君 and 民 belong to the same class - the same class of objects, political objects that is, that can occupy the position of 君. So in a sense, we can write $\text{君} \sim_{?} \text{民}$. Then, by simply substituting 君 for 民, we get 

$$
\text{\underline{民}主之國,萬方有罪,罪在\underline{君}身} \sim_{?} \text{\underline{民}主之國,兆方有罪,罪在\underline{民}身}
$$

Which says, "in a country where the people are sovereign, whoever is sinful, the guilt is on the people's body." 



This is not a derivation. Or is it? What kind of derivation is this? What kind of rule of inference are we appealing to? And what is the argument sayiing? Are we saying if (君主之國,萬方有罪,罪在朕躬) and $\text{君} \sim_{?} \text{民}$ then we can derive (民主之國,萬方有罪,罪在民身)? What laws govern the behaviour of the $\sim_{?}$? Are they the same as the laws governing the behaviour of $\sim$?

\subsection{The Core Behaviour of Synonym Quotient Siumatgwoons}

One core behaviour governing a siumatgwoon with synonymns, and its associated synonym quotient siumatgwoon ($S/\{\sim_i\}$) is that

$$[a] = [b] \iff a\sim_i b \text{ for some } \sim_i$$

In my mind, the way I think about the $\iff$ symbol is that $\phi \iff \psi$ says $\phi \models \psi$ and $\psi \rmodels \phi$. 

With this armed in hand, can we now explore the world of how synonyms can be exchanged in composites? Such how atomics may be switched in elementals and compounds, how prefixes and suffixes may be switched in words, how words may be switched in sentences, and how sentences may be switched in paragraphs, and so on. 

One promising candidate seems to be this: if $[ab]=[cd]$ and $a\sim x$ then $[ax] = [cd]$.

But this introduces some problems on the $\sim$ relation(s) involved. Let us explicate what $[ab] = [cd]$ says. $[ab] = [cd]$ means $ab \sim_i cd$ for some $\sim_i$. Suppose $a \sim_j x$, given that we have two, not necessarily identitcal synonym relations $\sim_i$ and $\sim_j$, well - what does it mean? Is the new $[ax]=[cd]$ governed by $\sim_i$ or $\sim_j$?

Naively, we may just let it pass, and assert and make an axiom out of this: 
$$ [ab] = [cd]_{(\sim_i)}, a\sim_j x \models [ax] = [cd] _{(\sim_i)} $$ 
%  mark this somehow 
Perhaps this suggests that the sentence, or the higher order $\sim$, which governs over higher order compounds rather than the lower order ones, takes precedence and dominance. This might be intuitively appealing. After all, it seems reasonable to say that if two words are synonymous then surely you can swap them out in a sentence. But just because two sentences are synonymous doesn't mean the words, or the subjects in the sentence are synonymous. Id est: 

$$\sim_j > \sim_i$$ 

Obviously, this gives us another something for us to potentially axiomatize and structuralise, which means there's another structure here to explore, and axioms for us to configure and callibrate - but let's ignore that for now. 








醫 毉 嫛
But doesn't this break in the case 醫 毉 嫛? From variant sinoglyphs, we know:

$$[\text{醫}]=[\text{毉}]=[\text{嫛}] $$

And the reason why this chain is admitted as plausible as per the siumatgwoon embedded in the Sinoglyphs is that in Sinitic metaphysics, women 女 and witches 巫 are synonymous, (in a sense, obviously). Hence, $\text{女}\sim_j\text{巫}$. And it is because of this synonymity, that writers, calligraphists, and the extended literati, have evolved to collectively grant legitimacy to this variant exchange symbolized by $[\text{醫}]=[\text{毉}]=[\text{嫛}]$.

What we want, is some kind of rule of inference that captures the gist of this proto-rule, this custom, this convention. 

Suppose we start with $$[\text{醫}]=[\text{毉}]$$
which means $\text{醫}\sim_i\text{毉}$, but since we also know $\text{女}\sim_j\text{巫}$, by the intuition above, we can then say that $\text{醫}\sim_i\text{毉} \sim_i \text{毉}\sim_i\text{巫}$, and so $\text{醫}\sim_i\text{嫛}$.

And 

and 

note that the 医殹醫毉嫛𮋺 set of variants is a very valuable example and test case for our axioms born out of intuitions. It is one of the few sinoglyphs with two operating semanophores, where one is switched out amongst variants. 


Now something trickier - but something we'd probably would very much want - is it possible to produce an axiom such that we can "cancel" out irrelevant components from synonymns so that the "irrelevant" components are eliminated, leaving us only the core synonymous components - that are in a sense responsible for the synonymn of the compound? Now this is very tricky. The sinoglyphs seem to give examples for and against this, sometimes it runs but others it doesn't - probably because the deep underlying aesthetic and associations were not fully graphically expressed. 

If we have to reason backwards as to why  $\text{女}\sim_j\text{巫}$ from $\text{醫}\sim_i\text{嫛}$, one might say that's because all the other components are the same in both cases, and so the only difference is the 女 and 巫. In the particular case of $\text{醫}\sim_i\text{嫛}$, they both share 医殳, the only difference is the 女 and 巫.

In a sense, one can perhaps treat the phonetic component, the name 殳 as an object in of itself. If you view it such a way, then you have

$$\text{醫}=\text{医}*\text{女}*\text{殳} = (\text{医}*\text{女})_{\text{殳}}$$

So, perhaps a promising axiom to capture this is: 

If $[x]=[y]$ and the components of $x$ is $X$ and that of $y$ is $Y$, and if $X - X \cup Y = Y - X \cup Y = \{a_1, a_2, \ldots, a_n\}$ then $a_1 \sim_i a_2 \sim_i \ldots \sim_i a_n$ for some $\sim_i$.



Basically whatever they do not share must synonymous. 

This proposal is applicable in $$\text{體} , \text{骵}, \text{躰}, \text{体}$$, which yields,
$$[\text{骵}]=[\text{躰}]=[\text{体}] \models [\text{骨}]=[\text{身}]=[\text{人}]$$
% 𲁢
And in $$\text{信},\text{訫},\text{𬢭},\text{⿰言千},\text{䛨},\text{孞},\text{伩},\text{⿰亻德}$$, it gives us:

$$[\text{信}]=[\text{訫}]=[\text{𬢭}] \models [\text{人}]=[\text{心}]=[\text{身}]$$

$$[\text{信}]=[\text{伩}]=[\text{⿰亻德}]\models [\text{言}]=[\text{文}]=[\text{德}]$$






 But it is not true that 千~人 nor 辛~人 or 辛~心. 
 So where do the chains end?
 Well, one slapstick way to explain it away is to say that the 言 in $\text{信},\text{訫},\text{𬢭}$ and the 言 in $\text{⿰言千},\text{䛨}$ are not the same - it's just a matter of writing convention that they're assigned the same glyph. "In reality", they are better symbolised as $\text{言}_1$ and $\text{言}_2$ respectively. 

 Another way is to perhaps observe that the poetic interpretation of $\text{信},\text{訫},\text{𬢭}$ emphasizes that "trust" 信 emerges from "speech" 言 that comen from "man", "heart", or "body", whereas in $\text{⿰言千},\text{䛨}$, it asserts "trust" 信 (or perhaps "trustworthiness") emerges from "speech" 言 that is  "a thousand" 千, or that is "bitter" 辛. Well, the two explanatory structures are not the same... so the factor elimination fails! 

 This is a naive justification - but perhaps again offers some kind of further structure for us to define and play around with. 

 
 
 
 
 
 
 
 
 
 
 








It also succeeds in 

信訫𬢭⿰𲁢䛨孞伩⿰亻德

Which would yeild 

法灋㳒佱𫼒