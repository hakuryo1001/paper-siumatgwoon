Consider the old siumatgwoon axioms, and a new relation "synonym", represented by $\sim$, read as "is synonymous with". It satisfies the following axioms.

\subsection{Synonym Axioms}

\begin{axiom}[Reflexivity]\label{ax:syn-reflex}
S1. (reflexivity) for all $x\in S$, $x\sim x$.
\end{axiom}

\begin{axiom}[Symmetry]\label{ax:syn-sym}
S2. (symmetry) For all $a,b\in S$, if $a\sim b$ then $b\sim a$.
\end{axiom}

\begin{axiom}[Transitivity]\label{ax:syn-trans}
S3. (transitivity) for all $a,b,c \in S$, if $a\sim b$, and $b \sim c$, then $a \sim c$.
\end{axiom}

\begin{axiom}[Compositional Congruence]\label{ax:syn-comp}
S4. (Compositional congruence) If $a \sim a'$, then (if they $ax$ or $xa$ exists):
\begin{itemize}
\item $ax \sim a'x$,
\item $xa \sim xa'$
\end{itemize}
\end{axiom}

\begin{axiom}[Composition Cancellation]\label{ax:syn-cancel}
S5. Composition cancellation
\begin{itemize}
\item If $a * b \sim a * c$, then $b \sim c$;
\item If $b*a\sim c*a$ then $b \sim c$;
\end{itemize}
\end{axiom}

\begin{axiom}[Divisor Compatibility]\label{ax:syn-div}
S6. Divisor compatibility: If $a \sim b$, then for all $x \in S$, $x|a$ iff $x|b$.
\end{axiom}

\begin{axiom}[Upwards Divisibility]\label{ax:syn-up}
S7: Upwards divisibility: If $a\sim b$, then for all if $a|x$ then $b|x$.
\end{axiom}

A Siumatgwoon is Synonym-Closed if in which if $a*b$ exists and $a\sim a'$ and $b\sim b'$ then $a'*b, a*b', a'*b'$ also exist. Most siumatgwoons, including the Sinoglyphs, are not synonym-closed. 

Because of S4, Synonym-Closed Siumatgwoons must have $a*b \sim a'*b \sim a*b'\sim a'*b'$.

\subsection{S8: Synonym Replacement and Existence Preservation}

Synonym replacement completion: if $a*b$ exists, and if $b\sim b'$, then $a*b'$ exists in $S$ as well.

Prior to this, we are rather ambiguous and coy as to whether $a$ and $b$ being synonymous means $b*x$ necessarily exists if $a*x$ exists. I feel we should admit closure for synonym substitutional compositions. We can consider other structure where this axiom is not admitted later.

S9. Multidecomposition if $ab \sim cd$ and $cd = e$ then $ab =e$. 

From S4 you'd be able to prove that $a \sim a'$ and $b\sim b'$ then $a * b \sim a' * b'$.

\section{The Quotient Siumatgwoon $S/\sim$}

\subsection{Starting Point: Synonymy $\sim$ as an Equivalence Relation}

Let $S$ be a Siumatgwoon, and let $\sim$ be a synonymy relation on $S$ satisfying:

\begin{itemize}
\item $\sim$ is an equivalence relation: reflexive, symmetric, transitive.
\item $\sim$ is \textbf{compatible with composition}:
    
    If $a\sim a'$, $b\sim b'$, then $a*b\sim a'*b'$.
    
\item $\sim$ is \textbf{compatible with divisibility}:
    
    If $a\sim b$, then $x|a \iff x|b$ for all $x\in S$.
\end{itemize}

These give us a solid foundation to define a quotient structure.

\subsection{The Set $S/\sim$}

Let:

$S/\sim=\{[a]:a\in S\}$

where $[a]=\{x\in S\mid x\sim a\}$ is the \textbf{equivalence class} of $a$.

\subsection{Defining the Operations on $S/\sim$}

\subsubsection{Multiplication}

We define:

$[a]*[b]:=[a*b]$, if $a*b$ exists.

This is well-defined \textbf{because of compositional congruence} (axiom S4).

It is also because that we required if $a*b$ exists, so must $a'*b'$.

That is: if $a \sim a'$, and $b \sim b'$, then:

$a*b\sim a'*b'\Rightarrow [a*b]=[a'*b']$

So the result is independent of the representative.

\subsubsection{Divisibility}

We define:

$[a]|[b] \iff a|b$

Again, this is \textbf{well-defined} thanks to the \textbf{divisibility equivalence} axiom (S6):

If $a\sim a', b \sim b'$, given we have the axiom that $a|b \iff a'|b'$ then:

$[a]|[b] \iff [a']|[b']$

Thus, $|$ descends to the quotient.

\subsection{Verifying Siumatgwoon Axioms in $S/\sim$}

Let's verify that the \textbf{quotient structure inherits} the original Siumatgwoon axioms:

\begin{itemize}
\item \textbf{Axiom 1 (Reflexivity of $|$)}:
    
    Since $a|a\Rightarrow[a]|[a]$
    
\item \textbf{Axiom 2 (Totality)}:
    
    For all $[a],[b]$, either $[a]|[b]$ or not.
    
    This follows since $|$ on $S$ has totality, and divisibility is preserved under $\sim$.
    
\item \textbf{Axiom 3 (Transitivity)}:
    
    Suppose $[a]|[b]$ and $[b]|[c]$. Recall the definition that $[a]|[b]$ then $a|b$. It's straightforward.
    
\item \textbf{Axiom 4 (Composition and divisibility)}:
    
    We want to prove that if $[a]*[b]=[c]$, then $[a],[b] | [c]$. By definition $[a]*[b]=[c]$ then $a*b\sim c$, and so $a|c$. By the definition of constitutiveness over synonym classes, $[a]|[c]$. The argument same goes for $[b]|[c]$.
\end{itemize}

Now come axioms 5 and axioms 6. Do they hold in $S/\sim$, even if they might not hold in $S$?

\begin{itemize}
\item Axiom 5 (Complete Constructibility and Generators): There exists a non-empty subset $G_{S} \subseteq S$ such that:
    \begin{itemize}
    \item (Generation) Every element $x \in S$ is a product of a finite sequence of elements from $G_S$, i.e., $x = g_1 * g_2 * \cdots * g_n$, where $g_1, \dots, g_n \in G_S$
    \end{itemize}

And a Simple Siumatgwoon is one where:

\begin{enumerate}
\item $E$ is a generating set, and 
\item that decompositions into elementals are unique.
\end{enumerate}

\item Axiom 6. (Finite constitution). For any $x \in S$, there are only finitely many objects $y_1, y_2, \ldots, y_n \in S$ such that $y_1, y_2, \ldots, y_n | x$.
\end{itemize}

Let's prove that axiom 5 holds first. And then axiom 6 holds. Then we prove that $S/\sim$ is a simple siumatgwoon.

Consider some element $[x] \in S/\sim$. Given $S$ is a siumatgwoon, we know that any $x$ has a finite decomposition $g_1 * g_2 * g_3 * \cdots g_n$. Given $x=g_1 * g_2 * g_3 * \cdots g_n$, we have $[x]=[g_1 * g_2 * g_3 * \cdots g_n]$, and by the definition of $[*]$ we have $[x]=[g_1 * g_2 * g_3 * \cdots g_n]=[g_1] * [g_2] * [g_3] * \cdots [g_n]$. Given this applies to every $x$ in $S$, it is true of every $[x]\in S/\sim$. 

Note that some of the $[g_i]$ might be equal to some other $[g_j]$ if $g_i \sim g_j$.

Now for axiom 6.

Consider $[x]\in S/\sim$. By axiom 6, there only finitely many $y_1, y_2, \ldots, y_n | x.$ Therefore there are only finitely many $[y_1], [y_2], \ldots, [y_n] | [x].$ Note again, that because there could be $y_i \sim y_j$, then the most reduced list of constituents for $[x]$ might be shorter than $n$.

\textbf{Axiom 5 (Generators)}:

If $G\subseteq S$ is a generating set, then $[G]=\{[g]\mid g\in G\}\subseteq S/\sim$ generates all of $S/\sim$

Hence, $S/\sim$ is a well-defined \textbf{quotient Siumatgwoon}.

\subsection{Interpretation of $S/\sim$}

\begin{itemize}
\item \textbf{Elements of $S/\sim$} are \textbf{semantic classes}: bundles of expressions that mean the same thing.
\item \textbf{Operations and structure} are all inherited from how the parts combine.
\item This is essentially a move from \textbf{syntax to semantics} — a form of canonical simplification.
\end{itemize}

\subsection{Identity and Atoms in $S/\sim$}

\begin{itemize}
\item The set of \textbf{atomic classes} would be:

$A/\sim=\{[a]\mid a\in A\}$

If synonymy is strong (e.g., if all synonyms of atoms are equal), then this is a \textbf{true set of atomic meanings}.

\item The quotient set \textbf{removes redundancy}: if two characters are written differently but function identically, they're collapsed.
\end{itemize}

\section{A Hierarchy of Synonym Relations}

Intuitively speaking, the $\sim$ of 言$\sim$文 that allows us write 信$\sim$伩 seems different from the $\sim$ in 女$\sim$巫, as it does not seem the case that 女 can be exchanged for 巫 in all Chinese characters. This suggests that there is a hierarchy, or rather, a family of synonymous relations, perhaps context dependent. 

Some synonyms are more closely clustered together than others. This suggests there is a hierarchy of synonyms.



But to interpret $~$ to only mean "is synonymous with", leaves us certain phenonomena in, or rather, certain ways to "play" Chinese characters, uncaptured. 

In Classical Chinese poetry and prose, it is appreciated if couplets are written in a way that characters in the same position belong to "some same class" or objects. 

Consider all these lines from 《增廣賢文》:


兩人一般心,無錢堪買金;一人一般心,有錢難買針。
馬行無力皆因瘦,人不風流只爲貧。

畫虎畫皮難畫骨,知人知面不知心。
錢財如糞土,仁義値千金。
流水下灘非有意,白雲出岫本無心。

路遙知馬力,日久見人心。
相見易得好,久住難爲人。
饒人不是癡漢,癡漢不會饒人。
有錢道眞語,無錢語不眞。
長江後浪推前浪,世上新人趕舊人。
來如風雨,去似微塵。
萬般皆是命,半點不由人。


寧可正而不足,不可邪而有餘。
寧可信其有,不可信其無。


To illustrate the point, consider this following sentence, taken from the Analects. 

萬方有罪,罪在朕躬。

"Whoever is sinful, the guilt is on my body."

This sentence clearly expresses a monarchist position of politics - one that is shared by both classical Confucianism and medieval Christian political theory. The sovereign, the 君主, is the one who bears the responsibility, and the sins of the 萬方.

Let us simplify and modify this sentence a bit: 

君主之國,萬方有罪,罪在君身。

"In a country where monarchs are sovereign, whoever is sinful, the guilt is on the monarch's body."

Now a we can apply a naive application of lax synonymy, say on 萬 ("ten thousand") and 兆 ("trillion"), drawing upon they can both be taken to mean "uncountably many" on a metaphoric level, so $\text{萬} \sim \text{兆}$, so we get 
$$
\text{君主之國,\underline{萬}方有罪,罪在君身}
\sim
\text{君主之國,\underline{兆}方有罪,罪在君身}
$$

We don't have the full algebraic machinery to justify this yet, but it is reasonable. 

What I really want to point out, is that in a sense, 君 and 民 belong to the same class - the same class of objects, political objects that is, that can occupy the position of 君. So in a sense, we can write $\text{君} \sim_{?} \text{民}$. Then, by simply substituting 君 for 民, we get 

$$
\text{\underline{民}主之國,萬方有罪,罪在\underline{君}身} \sim_{?} \text{\underline{民}主之國,兆方有罪,罪在\underline{民}身}
$$

Which says, "in a country where the people are sovereign, whoever is sinful, the guilt is on the people's body." 



This is not a derivation. Or is it? What kind of derivation is this? What kind of rule of inference are we appealing to? And what is the argument sayiing? Are we saying if (君主之國,萬方有罪,罪在朕躬) and $\text{君} \sim_{?} \text{民}$ then we can derive (民主之國,萬方有罪,罪在民身)? What laws govern the behaviour of the $\sim_{?}$? Are they the same as the laws governing the behaviour of $\sim$?

