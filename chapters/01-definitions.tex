\section{Definitions}

\epigraph{今也,南蠻鴃舌之人,非先王之道。}{——《孟子.滕文公上》}


\section{A Foray into the Axioms of the Siumatgwoon}


\begin{definition}[Siumatgwoon]\label{def:siumatgwoon}
    A Siumatgwoon, or a Metaphysic, is a set $S$, paired with the binary relations \textit{composition} $* : S \times S \rightarrow S$ and \textit{constitution} $|:S\times S \rightarrow \{\text{True}, \text{False}\}$, such that the following axioms hold. 

    \begin{axiom}[Reflexivity]\label{ax:reflex}
    For all $a \in S$, $a|a$.
    \end{axiom}

    \begin{axiom}[Totality]\label{ax:total}
    For all $a,b\in S$, exactly one of the following holds: $a|b$ or not $a\not|b$
    \end{axiom}

    \begin{axiom}[Transitivity]\label{ax:trans}
    For all $a,b,c\in S$, $a|b$ and $b|c$ implies $a|c$.
    \end{axiom}

    \begin{axiom}[Composition Constitution]\label{ax:comp-const}
    If $a*b=c$ for some $a,b,c\in S$, then $a|c$ and $b|c$.
    \end{axiom}
\end{definition}

$a * b$ should be read is "$a$ composing with $b$", and $a | b$ should be read as "$a$ constitutes $b$". You can also read $*$ as "and", "along with", "mixed with", "reacting with", "making love with", etc. You can also read $|$ as "is a part of", "composes", "is a constituent of", "is a component of", "is a sub-object of", etc. Because we are exploring, we will enable maximal leeway in the interpretation of $*$ and $|$, and this we will see, can yield some happy and interesting structures. 

Some further clarifications: 

\begin{itemize}
\item $a\not|b$ iff $a|b = \text{false}$
\end{itemize}

The above 4 axioms are the core axioms of a Siumatgwoon. There are quite a lot of ways in which you can think about what they are. You can think of them as sure foundations you can rest upon and then launch to explore the space, or you can think of them as a set of definitions that define the metaphysics of an object, in this case a mathematical object. But all mathematical objects are metaphysical objects (though we don't know if all metaphysical objects are mathematical objects - I'd wager not - the first go-to reason one might want to appeal to is Godel's Incompleteness Theorem - I think that's the right direction but Godel won't give us the direct proof to support this intuition - it is likely going to be something beyond mathematical language. Indeed, I think there are metaphysical objects that can be described by natural language, but not mathematical language. But I digress). We can further define other types of Siumatgwoons with further axioms. A taxonomy would then emerge, defined by various properties captured by the additional axioms.

But to capture the logic inside the Sinoglyphs, which is where we are starting - and we are starting there because we believe there's an interesting logic that evolved in there - just like interesting creatures and biosphere usually evolve in isolated and unique environments. I'To specify certain more restrictive structures, we will now introduce two more axioms. 
These two axioms seem to capture certain behaviours of the Sinoglyphs, and are therefore interestingto 


\begin{definition}[Simple Siumatgwoon]\label{def:simple}
A Simple Siumatgwoon $S$ is a siumatgwoon where:
\begin{axiom}[Complete Constructibility and Generators]\label{ax:finite_decomposition} 
    There exists a non-empty subset $G_{S} \subseteq S$ such that every element $x \in S$ is a product of a finite sequence of elements from $G_S$, i.e., $x = g_1 * g_2 * \cdots * g_n$, where $g_1, \dots, g_n \in G_S$
    \end{axiom}
    
    \begin{axiom}[Finite Constitution]\label{ax:finite_constitution} 
        For any $x \in S$, there are only finitely many objects $y_1, y_2, \ldots, y_n \in S$ such that $y_1, y_2, \ldots, y_n | x$.
    \end{axiom}
    
\end{definition}

I stress finite. The point of these two axioms - is to ground ourselves in the world of finite compositions and finite constitutions. It will be very interesting to see what kind of objects will be yielded if we relax these two axioms - to infinite compositions and infinite constitutions. How will they look like? 



Well, if we relax \ref{ax:finite_decomposition}, one can have a siumatgwoon $$S$$ where everything can be infinitely decomposed. So there are no atoms - everything can be cut smaller and smaller and smaller - without end. 

Nowf we relax \ref{ax:finite_constitution}, so you can arguably create a siumatgwoon where everything is made of infinitely many objects, including the objects constituting other objects. 

These are some really weird objects indeed. But they are definitely not anything like the Sinoglyphs - at least not right now. 


Does the size of the infinity make things even more interesting? Careful now, we are trying to do what Leibniz didn't do, not to roleplay Cantor. 

Before we discuss some firther definitions, of Elementals, Generators, Atomics, let us remark some pattens we see in the Sinoglyphs. 

The first thing to remark is that Westerners have translated the 邊旁 into "radicals" - as if they're solutions to a quadratic equation, which means then they thought the whole sinoglyph itself is like an equation to be solved. 

If we take these radicals, that compose a sinoglyph, we'd say that they are the elementals of the sinoglyph. 

The way I came up with the set of definitions for atoics, elementals, generators - was to model the Sinoglyph writing system. Oftentimes, in the Sinoglyph system, as it appears to us, all three sets are the same. 

But if we consider a possible extension to the Chinese Character writing system, and simply consider the idealised and fully evolutionarily searched space of possible evolutionary trajectories of the Chinese character writing system, we can arrive at an idealised symbol combination system - that's the Sinoglyph. That Sinoglyph symbol manipulation system, has a structure. And that structure is its Siumatgwoon. And we are now trying to model that structure with some axioms. 

Atomics are basically 獨體字. They stand alone. They cannot be decomposed, and nothing constitutes them. 

Elementals are basically any sinolgyph that cannot be fully decomposed. Something else might constitute it, but its constitutes do not make whole the elemental. There's something more in the elemental than just the just whatever constituting it is. An elemental might have many many constituents, but the elemental is somehow larger than the composition of its parts. This can be called emergence. 

But not all elementals exhibit emergence. Because all atomics are elementals. Since atomics are constituted by themselves, and the atomic needs nothing to be composed with to be itself, atomics are elementals with no emergence. 

The elementals largely correspond to the kind of compound sinoglyphs 合體字 that are called phonosemanophores - 形聲字。 

Now let's think about how a 形聲字 works. Let's take the classic example of 江河, specifically 河. The Chinese character 河 is a 形聲字. 
$$\text{河}$$
It is composed of 氵(water) and 可 (ho). Ho is just a name. You might as well right it like this $$\text{氵}_{\text{可}}$$. But the sound is immaterial, it's just an index. It might as well be $a,b,c$

$$\text{氵}_{a},\text{氵}_b,\text{氵}_c$$
But the important thing is that picks something out. It picks out from the set of all objects that are constituted by water the thing that is "river". You might as very well write it as $$\text{氵}_{river}$$. 

What this immediately suggest is a 系 system. A 金木水火土系 system. A system where elementals can be long to 金系, 木系, 水系, 火系 土系 system is a Siumatgwoon, where each 系 contains (but is not limited to) elementals that are constituted of the well, the "Head" of the 系. So the head, of a 土系 is 土. The head of a 金系 is 金. The head of a 木系 is 木. The head of a 水系 is 水. The head of a 火系 is 火. 

We want to capture this - the set of all objects exhibiting emergence from a certain "head". Let us call this set of head x the x-系. 


\begin{definition}[x-系]\label{def:x-system}
    An x-系, written $\langle x \rangle$, is the set of all elementals that are constituted by the head $x$.
\end{definition}

Being elementals, they cannot be fully decomposed. So what's in an x-系?Say, 木系 $\langle \text{木} \rangle$?

Well, it contains $\text{木}$, because $\text{木}$ is cannot be further decomposed. But we also have the other trees, 松,柏,柳,桃,梅, so we can say 

$$\text{木}_{\text{公}},\text{木}_{\text{白}},\text{木}_{\text{卯}},\text{木}_{\text{兆}},\text{木}_{\text{每}},  \text{木} \in \langle \text{木} \rangle$$



But note that $\text{林}, \text{森},  \text{焚} \not\in \langle \text{木} \rangle$



Does the head of a 系 have to be an atomic element? No, not at all! In fact, the head of a 系 could itself be an elemental like 河, or a proper compound like 雲. Consider the siumatgwoon $S = \{\text{雨}, \text{云}, \text{雲}, \text{霒}, \text{霕}, \text{霴}, \text{靉}\}$. In this set clearly the atoms are $\{\text{雨}, \text{云}\}$, but the elementals are $\{\text{雨}, \text{云}, \text{霒}, \text{霕}, \text{霴}, \text{靉}\}$. So $\{\text{霒}, \text{霕}, \text{霴}, \text{靉}\}$ are all in $\langle \text{雨} \rangle$ and in $\langle \text{云} \rangle$ and also in $\langle \text{雲} \rangle$.

But there's no reason why elementals, 形聲字, themselves, cannot be the head of a 系. Why shouldn't there be say an object $x=\text{河}_{a}$ or $river_{a}$ that we know is a kind of river, but we know what all the other constituents are to make it more than just a river? I mean, it's pretty obvious that The Thames, La Seine, The Hudson, The Jyugong, are all rivers, i.e.: 
$$\text{河}_{Thames},\text{河}_{Seine},\text{河}_{Hudson},\text{河}_{\text{珠江}} \in \langle \text{河} \rangle$$
but we can't write down all the other constituents that make them more than just a river, and not how they differ from each other either.




The Chinese way of writing "river" is not different from writing it as "氵river". It says that "river" belongs to the class of things that are constituted of "氵"(water); it says "river" is a "氵" thing; it says "river" is belongs to the set, the class, the category of "氵" things; but most importantly, it says: I don't know how "river" is composed, but I do know it is constituted of "氵". So there's something beyond water that makes a river - but I don't know what all the stuff that composes a river is, or hanc marginis exiguitas non caperet.







\begin{definition}[Atomics]\label{def:atomics}
    An object $s$ in a Siumatgwoon $S$ is an atomic if it cannot be decomposed into smaller elements, i.e., if $s = a * b$ implies that either $a = s$ or $b = s$.
\end{definition}




\begin{definition}[Elementals]\label{def:elementals}
    The elementals of a Siumatgwoon $S$ are the elements that cannot be decomposed into smaller elements.
\end{definition}


\begin{definition}[Atomic Element]\label{def:atomic-element}
An element $a \in S$ is atomic if for all $x\in S$, if $x|a$ implies $x=a$, i.e. $a$ is atomic iff only itself constitutes itself.
\end{definition}

\begin{definition}[Atomic Set]\label{def:atomic-set}
The set $S_A \subseteq S$ is called the atomic set of $S$. It is the set of all atomic elements inside $S$. Where confusion does not exist, one could simply write $A$.
\end{definition}

\begin{definition}[Hai Elements]\label{def:hai-elements}
The set $\langle a \rangle := \{ x\in S: a|x \text{ but } \not \exists y \in S \text{ such that } ay=x\}$ is called the Hai elements of a, i.e. the $a$-系 elements. These correspond to the 形聲字 in Chinese characters - if you do not consider the phonetic component of a character to be full elements inside the Siumatgwun but mere indices. In other words, one does not view 江、河、湖、海 as 水工、水可、水胡、水每 but as $水_工、水_可、水_胡、水_每$. In this sense, one can see there is no element $y \in 字$ such that $水y=水_工=江$.
\end{definition}


Now we come to our first Theorem: 
\begin{theorem}
    Let $S$ be a Simple Siumatgwoon. And let $E$ be the set of elementals of $S$. Then: 
    \begin{enumerate}
        \item $E$ is a generating set, and 
        \item The decompositions into elementals are unique.
    \end{enumerate}
\end{theorem}
\begin{proof}
    Let $S$ be a Simple Siumatgwoon. And let $E$ be the set of elementals of $S$. Then: 
    Let's first prove that E is a generating set first.
    \begin{enumerate}
        \item $E$ is a generating set, and 
        \item The decompositions into elementals are unique.
    \end{enumerate}
\end{proof}


\subsection{Examples of Siumatgwoons}

\subsubsection{The Chinese Characters, 字}

The Chinese characters, which inspired this whole mathematical exercise, is clearly a Siumatgwoon. If we exercise the synonym exchange of "Siumatgwoon" with "Metaphysic", this is to say, that the Chinese characters is a Metaphysic. Unfortunately we can't really \textit{prove} that the Chinese characters are indeed a siumatgwoon, given there's an infinite number of them, and we do not have a generating rule for all Chinese characters. However, the fact that any subset of Chinese characters is a siumatgwoon in of itself, lends us confidence - perhaps there's a theorem there waiting to be proved?

\subsubsection{The Roman Numerals $\mathfrak{R}$}

One can clearly see that Roman Numerals $\mathfrak{R}$ are a Siumatgwoon. However, to appreciate the characteristics that make it a Siumatgwoon, let us consider the subset of Roman Numerals from 1 to 10, which we shall show to also be a Siumatgwoon.

$\mathfrak{R}_{1,10} = \{I, II, III, IV, V, VI, VII, VIII, IX, X\}$

We will say that for two elements $a,b \in \mathfrak{R}_{1,10}$, $a|b$ iff the glyph $a$ appears in $b$. As such, we can say $I | II$ and $I|III$ as an example, and that $V|IV$ and $X|IX$. 

For any elements $a,b\in \mathfrak{R}_{1,10}$, if the glyphs $ab$ so written together forms a glyph that also appears in $\mathfrak{R}_{1,10}$, then we'd say that $a*b\in \mathfrak{R}$.

Now, it's clear that Ax 1 is satisfied trivially. 

Ax 2 is also satisfied trivially.

Ax 3 is also satisfied. 

Ax 4 is also satisfied. As an example: $I | III, III | VIII$ and we have $I|VIII$.

So therefore, $\mathfrak{R}_{1,10}$ is a Siumatgwoon. 

It is also interesting to note that as per the definition of $\mathfrak{R}_{1,10}$, it is not compositionally closed. For example, $II * III$ is not in $\mathfrak{R}_{1,10}$. This makes the Siumatgwoon different from a group, where all compositions are contained inside the group. Intuitively, perhaps this suggests the Siumatgwoon is less rich in structure than the mathematical group? Also, note that what $II * III$ should be in $\mathfrak{R}_{1,10}$ is represented by $V$. Intuitively, we can feel that in some sense, $II * III = V$ - that they're synonymous, identical, referring to the same referent. This is not unlike the presence of variant characters in the Sinoglyphs, such as 體 (body, object)=骵=躰=体, or信 (trust) = 𬢭 = 伩 = 訫 = 㐰… Intuition should hint that this will yield some interesting structures if we pursue the investigation down this path.

\subsubsection{Any Numerals System}

The fact that the Roman Numerals are a Siumatgwoon should intuitively suggest that any numeral system is a Siumatgwoon. In fact, let us consider the world's many numeral systems, and see if there is one where it is not a siumatgwoon. 

\begin{center}
\begin{tabular}{|l|c|c|c|c|c|c|c|c|c|c|}
\hline
 & 0 & 1 & 2 & 3 & 4 & 5 & 6 & 7 & 8 & 9 \\
\hline
唐字數字 & 〇 & 一 & 二 & 三 & 四 & 五 & 六 & 七 & 八 & 九 \\
\hline
唐字數字大寫 & 零 & 壹、弌 & 貳 & 叄 & 肆 & 伍 & 陸 & 柒 & 捌 & 玖 \\
\hline
字喃 &  & 𠬠 & 𠄩 & 𠀧 & 𦊚 & 𠄼 & 𦒹 & 𦉱 & 𠔭 & 𠃩 \\
\hline
蘇州碼子 & 〇 & 〡、一 & 〢、二 & 〣、三 & 〤 & 〥 & 〦 & 〧 & 〨 & 〩 \\
\hline
Roman Numerals &  & I & II & III & IV & V & VI & VII & VIII & IX \\
\hline
Eastern Arabic & ٠ & ١ & ٢ & ٣ & ٤ & ٥ & ٦ & ٧ & ٨ & ٩ \\
\hline
Persian & ٠ & ۰ & ۱ & ۲ & ۳ & ۴ & ۵ & ۶ & ۷ & ۸ \\
\hline
Devanagari & ० & १ & २ & ३ & ४ & ५ & ६ & ७ & ८ & ९ \\
\hline
Gujarati & ૦ & ૧ & ૨ & ૩ & ૪ & ૫ & ૬ & ૭ & ૮ & ૯ \\
\hline
Tibetan & ༠ & ༡ & ༢ & ༣ & ༤ & ༥ & ༦ & ༧ & ༨ & ༩ \\
\hline
Hebrew &  & א & ב & ג & ד & ה & ו & ז & ח & ט \\
\hline
Chinese counting rods &  & 𝍠 & 𝍡 & 𝍢 & 𝍣 & 𝍤 & 𝍥 & 𝍦 & 𝍧 & 𝍨 \\
\hline
counting 正 &  & 𝍲 & 𝍳 & 𝍴 & 𝍵 & 𝍶 & 𝍶𝍲 & 𝍶𝍳 & 𝍶𝍴 & 𝍶𝍵 \\
\hline
Tangut &  & 𘈩 & 𗍫 & 𘕕 & 𗥃 & 𗏁 & 𗤁 & 𗒹 & 𘉋 & 𗢭 \\
\hline
\end{tabular}
\end{center}

I don't think there's a single one that's not a siumatgwoon! Most of them are pathological for sure, in the sense that nothing is constituted by anything else, but none of them violate the Siumatgwoon axioms! 

The case of the numerals as a Siumatgwoon, or a Metaphysic, is interesting. Numerals all refer to the same referents, the same "things" or "objects", namely, numbers. However, the glyphs in a given numeral system are themselves imbued with a particular set of metaphysical prejudices and judgements. Under the Roman Numeral Metaphysic, the number 3 is composed of 1 and 2, or composed of three 1s. 4 is composed of 1 and 5, but not 3 and 2. 

\subsubsection{The Polygons $\mathcal{P}$}

Consider the following graph. If we take all the polygons, convex and star, as elements in a set called $\mathcal{P}$, we can see that it forms a Siumatgwoon. We state this without formal proof for the infinite set $\mathcal{P}$, but from the subset displayed in the graph below, we can see it is indeed true. A polygon $a$ constitutes polygon $b$ if $a$ appears in $b$. $\{3\}$, the equilateral triangle, appears in $\{6/2\}$, the star of David, and so $\{3\} |\{6/2\}$

The Schläfli symbol is a recursive description, starting with $\{p\}$ for a $p$-sided regular polygon that is convex. For example, $\{3\}$ is an equilateral triangle, $\{4\}$ is a square, $\{5\}$ a convex regular pentagon, etc.

Regular star polygons are not convex, and their Schläfli symbols take the form $\{p/q\}$, where $p$ is the number of vertices and $q$ is their turning number. Equivalently, $\{p/q\}$ is created from the vertices of $\{p\}$ by connecting every $q$th vertex. For example, $\{5/2\}$ is a pentagram, while $\{5\}$ is a pentagon.

Note that $p$ and $q$ must be coprime, or the figure will degenerate, in which case we have the following theorem:

$\{p/q\}=d\{ \frac{p}{d} / \frac{q}{d} \}$, where $d=\gcd(p,q)$.

Let us define for any Schläfli symbol $\{p\} | n\{p\}$ for any $n$. It is intuitively true.

Then clearly axiom 1 is satisfied. Axiom 2 is also satisfied.

\subsubsection{Propositional Logic}

There are 3 flavors of $*$ in Propositional Logic: $\wedge, \vee, \rightarrow$. 

And we define $\phi | \psi$ if $\phi$ appears in $\psi$, for any wff $\phi, \psi$.

Then clearly all 4 of the core siumatgwoon axioms are satisfied. 

\begin{enumerate}
\item Trivial that all $\phi | \phi$
\item Also trivial that for any $\phi, \psi$, either $\phi|\psi$ or $\phi\not|\psi$.
\item Trivial as well that for any $\phi,\psi | \phi * \psi$.
\item Also trivial that if $\phi | \psi$ and $\psi | \theta$ then $\phi | \theta$.
\end{enumerate}

Propositional Logic as a Siumatgwoon has multiple interesting properties: 

\begin{enumerate}
\item It is compositionally complete. 
\item It is also constitutionally complete. 
\item Is it a simple siumatgwoon? Are decompositions finite? Yes. Are decompositions unique? Yes, up to reordering. So yes, it's a simple siumatgwoon.


\end{enumerate}

Consider the old siumatgwoon axioms, and a new relation "synonym", represented by $\sim$, read as "is synonymous with". It satisfies the following axioms.

\subsection{Synonym Axioms}

\begin{axiom}[Reflexivity]\label{ax:syn-reflex}
S1. (reflexivity) for all $x\in S$, $x\sim x$.
\end{axiom}

\begin{axiom}[Symmetry]\label{ax:syn-sym}
S2. (symmetry) For all $a,b\in S$, if $a\sim b$ then $b\sim a$.
\end{axiom}

\begin{axiom}[Transitivity]\label{ax:syn-trans}
S3. (transitivity) for all $a,b,c \in S$, if $a\sim b$, and $b \sim c$, then $a \sim c$.
\end{axiom}

\begin{axiom}[Compositional Congruence]\label{ax:syn-comp}
S4. (Compositional congruence) If $a \sim a'$, then (if they $ax$ or $xa$ exists):
\begin{itemize}
\item $ax \sim a'x$,
\item $xa \sim xa'$
\end{itemize}
\end{axiom}

\begin{axiom}[Composition Cancellation]\label{ax:syn-cancel}
S5. Composition cancellation
\begin{itemize}
\item If $a * b \sim a * c$, then $b \sim c$;
\item If $b*a\sim c*a$ then $b \sim c$;
\end{itemize}
\end{axiom}

\begin{axiom}[Divisor Compatibility]\label{ax:syn-div}
S6. Divisor compatibility: If $a \sim b$, then for all $x \in S$, $x|a$ iff $x|b$.
\end{axiom}

\begin{axiom}[Upwards Divisibility]\label{ax:syn-up}
S7: Upwards divisibility: If $a\sim b$, then for all if $a|x$ then $b|x$.
\end{axiom}

A Siumatgwoon is Synonym-Closed if in which if $a*b$ exists and $a\sim a'$ and $b\sim b'$ then $a'*b, a*b', a'*b'$ also exist. Most siumatgwoons, including the Sinoglyphs, are not synonym-closed. 

Because of S4, Synonym-Closed Siumatgwoons must have $a*b \sim a'*b \sim a*b'\sim a'*b'$.

\subsection{S8: Synonym Replacement and Existence Preservation}

Synonym replacement completion: if $a*b$ exists, and if $b\sim b'$, then $a*b'$ exists in $S$ as well.

Prior to this, we are rather ambiguous and coy as to whether $a$ and $b$ being synonymous means $b*x$ necessarily exists if $a*x$ exists. I feel we should admit closure for synonym substitutional compositions. We can consider other structure where this axiom is not admitted later.

S9. Multidecomposition if $ab \sim cd$ and $cd = e$ then $ab =e$. 

From S4 you'd be able to prove that $a \sim a'$ and $b\sim b'$ then $a * b \sim a' * b'$.

\section{The Quotient Siumatgwoon $S/\sim$}

\subsection{Starting Point: Synonymy $\sim$ as an Equivalence Relation}

Let $S$ be a Siumatgwoon, and let $\sim$ be a synonymy relation on $S$ satisfying:

\begin{itemize}
\item $\sim$ is an equivalence relation: reflexive, symmetric, transitive.
\item $\sim$ is \textbf{compatible with composition}:
    
    If $a\sim a'$, $b\sim b'$, then $a*b\sim a'*b'$.
    
\item $\sim$ is \textbf{compatible with divisibility}:
    
    If $a\sim b$, then $x|a \iff x|b$ for all $x\in S$.
\end{itemize}

These give us a solid foundation to define a quotient structure.

\subsection{The Set $S/\sim$}

Let:

$S/\sim=\{[a]:a\in S\}$

where $[a]=\{x\in S\mid x\sim a\}$ is the \textbf{equivalence class} of $a$.

\subsection{Defining the Operations on $S/\sim$}

\subsubsection{Multiplication}

We define:

$[a]*[b]:=[a*b]$, if $a*b$ exists.

This is well-defined \textbf{because of compositional congruence} (axiom S4).

It is also because that we required if $a*b$ exists, so must $a'*b'$.

That is: if $a \sim a'$, and $b \sim b'$, then:

$a*b\sim a'*b'\Rightarrow [a*b]=[a'*b']$

So the result is independent of the representative.

\subsubsection{Divisibility}

We define:

$[a]|[b] \iff a|b$

Again, this is \textbf{well-defined} thanks to the \textbf{divisibility equivalence} axiom (S6):

If $a\sim a', b \sim b'$, given we have the axiom that $a|b \iff a'|b'$ then:

$[a]|[b] \iff [a']|[b']$

Thus, $|$ descends to the quotient.

\subsection{Verifying Siumatgwoon Axioms in $S/\sim$}

Let's verify that the \textbf{quotient structure inherits} the original Siumatgwoon axioms:

\begin{itemize}
\item \textbf{Axiom 1 (Reflexivity of $|$)}:
    
    Since $a|a\Rightarrow[a]|[a]$
    
\item \textbf{Axiom 2 (Totality)}:
    
    For all $[a],[b]$, either $[a]|[b]$ or not.
    
    This follows since $|$ on $S$ has totality, and divisibility is preserved under $\sim$.
    
\item \textbf{Axiom 3 (Transitivity)}:
    
    Suppose $[a]|[b]$ and $[b]|[c]$. Recall the definition that $[a]|[b]$ then $a|b$. It's straightforward.
    
\item \textbf{Axiom 4 (Composition and divisibility)}:
    
    We want to prove that if $[a]*[b]=[c]$, then $[a],[b] | [c]$. By definition $[a]*[b]=[c]$ then $a*b\sim c$, and so $a|c$. By the definition of constitutiveness over synonym classes, $[a]|[c]$. The argument same goes for $[b]|[c]$.
\end{itemize}

Now come axioms 5 and axioms 6. Do they hold in $S/\sim$, even if they might not hold in $S$?

\begin{itemize}
\item Axiom 5 (Complete Constructibility and Generators): There exists a non-empty subset $G_{S} \subseteq S$ such that:
    \begin{itemize}
    \item (Generation) Every element $x \in S$ is a product of a finite sequence of elements from $G_S$, i.e., $x = g_1 * g_2 * \cdots * g_n$, where $g_1, \dots, g_n \in G_S$
    \end{itemize}

And a Simple Siumatgwoon is one where:

\begin{enumerate}
\item $E$ is a generating set, and 
\item that decompositions into elementals are unique.
\end{enumerate}

\item Axiom 6. (Finite constitution). For any $x \in S$, there are only finitely many objects $y_1, y_2, \ldots, y_n \in S$ such that $y_1, y_2, \ldots, y_n | x$.
\end{itemize}

Let's prove that axiom 5 holds first. And then axiom 6 holds. Then we prove that $S/\sim$ is a simple siumatgwoon.

Consider some element $[x] \in S/\sim$. Given $S$ is a siumatgwoon, we know that any $x$ has a finite decomposition $g_1 * g_2 * g_3 * \cdots g_n$. Given $x=g_1 * g_2 * g_3 * \cdots g_n$, we have $[x]=[g_1 * g_2 * g_3 * \cdots g_n]$, and by the definition of $[*]$ we have $[x]=[g_1 * g_2 * g_3 * \cdots g_n]=[g_1] * [g_2] * [g_3] * \cdots [g_n]$. Given this applies to every $x$ in $S$, it is true of every $[x]\in S/\sim$. 

Note that some of the $[g_i]$ might be equal to some other $[g_j]$ if $g_i \sim g_j$.

Now for axiom 6.

Consider $[x]\in S/\sim$. By axiom 6, there only finitely many $y_1, y_2, \ldots, y_n | x.$ Therefore there are only finitely many $[y_1], [y_2], \ldots, [y_n] | [x].$ Note again, that because there could be $y_i \sim y_j$, then the most reduced list of constituents for $[x]$ might be shorter than $n$.

\textbf{Axiom 5 (Generators)}:

If $G\subseteq S$ is a generating set, then $[G]=\{[g]\mid g\in G\}\subseteq S/\sim$ generates all of $S/\sim$

Hence, $S/\sim$ is a well-defined \textbf{quotient Siumatgwoon}.

\subsection{Interpretation of $S/\sim$}

\begin{itemize}
\item \textbf{Elements of $S/\sim$} are \textbf{semantic classes}: bundles of expressions that mean the same thing.
\item \textbf{Operations and structure} are all inherited from how the parts combine.
\item This is essentially a move from \textbf{syntax to semantics} — a form of canonical simplification.
\end{itemize}

\subsection{Identity and Atoms in $S/\sim$}

\begin{itemize}
\item The set of \textbf{atomic classes} would be:

$A/\sim=\{[a]\mid a\in A\}$

If synonymy is strong (e.g., if all synonyms of atoms are equal), then this is a \textbf{true set of atomic meanings}.

\item The quotient set \textbf{removes redundancy}: if two characters are written differently but function identically, they're collapsed.
\end{itemize}

\section{A Hierarchy of Synonym Relations}

Intuitively speaking, the $\sim$ of 言$\sim$文 that allows us write 信$\sim$伩 seems different from the $\sim$ in 女$\sim$巫, as it does not seem the case that 女 can be exchanged for 巫 in all Chinese characters. This suggests that there is a hierarchy, or rather, a family of synonymous relations, perhaps context dependent. 

Some synonyms are more closely clustered together than others. This suggests there is a hierarchy of synonyms.
