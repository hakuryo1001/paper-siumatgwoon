\documentclass[9pt]{article}

\usepackage{amsmath,amssymb}
\usepackage[utf8]{inputenc}
\usepackage{ctex}
\usepackage{epigraph}
\usepackage{amsthm}

% Remove paragraph indentation for cleaner look
\setlength{\parindent}{0pt}
\setlength{\parskip}{1em}

% Reduce line spacing for more compact text
\linespread{0.2}

% Define theorem environments
\newtheorem{theorem}{Theorem}
\newtheorem{lemma}[theorem]{Lemma}
\newtheorem{corollary}[theorem]{Corollary}
\newtheorem{proposition}[theorem]{Proposition}
\newtheorem{definition}[theorem]{Definition}
\newtheorem{example}[theorem]{Example}
\newtheorem{remark}[theorem]{Remark}
\newtheorem{axiom}[theorem]{Axiom}


% Set CJK main font (for Chinese/Japanese/Korean characters)

% \setmainfont{Times New Roman}
\setCJKmainfont{BabelStone Han}
% doesn't work
% \setCJKmainfont{JyutcitziWithSourceHanSerifTCRegular}[
% Renderer=Basic,
% UprightFont = * ,
% FallbackFonts={BabelStone Han}
% ]



% You can also use \newfontfamily for custom non-CJK fonts if needed
% \setCJKmainfont{JyutcitziWithPMingLiURegular}[Path = ./, Extension = .ttf]
% \setCJKmainfont{JyutcitziWithSourceHanSerifTCRegular}[Path = ./, Extension = .ttf]



\newfontfamily{\jczPMingLiU}{JyutcitziWithPMingLiURegular}[Path = ./fonts/, Extension = .ttf]
% This has the best rendition for latin characters 
\newfontfamily{\jcz}{JyutcitziWithSourceHanSerifTCRegular}[Path = ./fonts/, Extension = .ttf]
\newfontfamily{\batang}{batang}[Path = ./fonts/, Extension = .ttf]
\newCJKfontfamily\koreanfont{Batang}[Path = ./fonts/, Extension = .ttf]
\newfontfamily{\taigi}{GentiumBookPlus-Regular}[Path = ./fonts/, Extension = .ttf]

% \newfontfamily{\biaoyinzi}{Biaoyinzi-2016A}[Path = ./fonts/, Extension = .ttf]

% Load ruby package for furigana (Ruby text)
\usepackage{ruby}

% IDC - ideographic description characters
% https://en.wikipedia.org/wiki/Chinese_character_description_languages#Ideographic_Description_Sequences

\newcommand{\superimpose}[2]{{%
  \ooalign{%
    \hfil$\m@th\text{#1}\@firstoftwo\text{#2}$\hfil\cr
    \hfil$\m@th\text{#1}\@secondoftwo\text{#2}$\hfil\cr
  }%
}}


% Define the \tb command

\newcommand{\tb}[2]{%
\scalebox{2}[1]{
\ooalign{%
    \hfil\raisebox{0.25em}{\text{\scalebox{0.33}{#1}}}\hfil\cr % Top text, squished and raised
    \hfil\raisebox{-0.25em}{\text{\scalebox{0.33}{#2}}}\hfil\cr % Bottom text, squished and lowered
  }%
  }
}

% The \lr command - can be combined with \tb
\newcommand{\lr}[2]{
  \scalebox{0.5}[1.0]{#1}\scalebox{0.5}[1.0]{#2}\!\!
}

% Define the \ul command for upper left positioning - for characters like 疒
\newcommand{\ul}[2]{%
  \ooalign{%
    \hfil#1\hfil\cr  % Top text (unscaled)
    \hfil\hspace{0.3em}\scalebox{0.8}{#2}\cr % Bottom text (scaled and raised)
    % \hfil\raisebox{0.2em}{\scalebox{0.5}{#2}}\hfil\cr % Bottom text (scaled and raised)
  }%
}

% Define the \tone command for upper right positioning of a diacritic
\newcommand{\tone}[2]{%
  \ooalign{%
    \hfil#1\hfil\cr  % Main text (unscaled)
    \hfil\hspace{0.9em}\raisebox{0.3em}{\scalebox{0.8}{#2}}\hfil\cr % Tone mark (scaled and raised)
  }%
}

% for vertical Chinese boxes
\usepackage{graphicx} % for \rotatebox

\newfontlanguage{Chinese}{CHN}

\setCJKfamilyfont{BabelStoneVert}[RawFeature={vertical;+vert},Script=CJK,Language=Chinese,Vertical=RotatedGlyphs]{BabelStone Han}

\newcommand*\CJKmovesymbol[1]{\raise.35em\hbox{#1}}
\newcommand*\CJKmove{\punctstyle{plain}% do not modify the spacing between punctuations
  \let\CJKsymbol\CJKmovesymbol
  \let\CJKpunctsymbol\CJKsymbol}

% Define a new environment for vertical text
\newcommand{\VertCell}[1]{\rotatebox{-90}{\CJKfamily{BabelStoneVert}\CJKmove #1}}


% *-----------------------------------------------------------------------*
% | Packages and formatting                                               |
% *-----------------------------------------------------------------------*
% *-----------------------------------------------------------------------*
% | Math & Equations     |
% *-----------------------------------------------------------------------*
\usepackage{amsmath} % For advanced math formatting
\usepackage{amssymb} % For mathematical symbols
% \usepackage{tikz} % For drawing logic decision trees


% *-----------------------------------------------------------------------*
% | Table Management                                                      |
% *-----------------------------------------------------------------------*


\usepackage{graphicx}
\usepackage{array}
\usepackage{tabularx}
\usepackage{tabularray}

\usepackage{float}      % Add the float package
\usepackage{longtable}


\usepackage[table,xcdraw]{xcolor}




% super long table

\usepackage{pdflscape} % For rotated pages

\setlength{\arrayrulewidth}{0.5mm} % Optional: to thicken table lines
\renewcommand{\arraystretch}{1.5} % Optional: to increase row height


\usepackage{makecell} 


% *-----------------------------------------------------------------------*
% | Chinese and Soochow Numerals                                          |
% *-----------------------------------------------------------------------*
% numerals.tex
% Define Chinese and Soochow numerals for chapter management

\newcommand{\soochowNumeral}[1]{%
  \ifnum#1<10
    \ifcase#1 〇\or 〡\or 〢\or 〣\or 〤\or 〥\or 〦\or 〧\or 〨\or 〩\fi%
  \else
    \ifnum#1<20
      〸\soochowUnits{\numexpr#1-10\relax}%
    \else
      \ifnum#1<30
        〹\soochowUnits{\numexpr#1-20\relax}%
      \else
        \ifnum#1<40
          〺\soochowUnits{\numexpr#1-30\relax}%
        \else
          \ifnum#1<50
            卅\soochowUnits{\numexpr#1-40\relax}%
          \else
            \ifnum#1<60
              〥十\soochowUnits{\numexpr#1-50\relax}%
            \else
              \ifnum#1<70
                〦十\soochowUnits{\numexpr#1-60\relax}%
              \else
                \ifnum#1<80
                  〧十\soochowUnits{\numexpr#1-70\relax}%
                \else
                  \ifnum#1<90
                    〨十\soochowUnits{\numexpr#1-80\relax}%
                  \else
                    \ifnum#1<100
                      〩十\soochowUnits{\numexpr#1-90\relax}%
                    \fi
                  \fi
                \fi
              \fi
            \fi
          \fi
        \fi
      \fi
    \fi
  \fi
}

\newcommand{\soochowUnits}[1]{%
  \ifnum#1=0
  \else
    \ifnum#1<4
      \ifcase#1 \or 一\or 二\or 三\fi%
    \else
      \soochowNumeral{#1}
    \fi
  \fi
}

\newcommand{\chinesenumeral}[1]{%
  \ifnum#1<10
    \ifcase#1 〇\or 一\or 二\or 三\or 四\or 五\or 六\or 七\or 八\or 九\fi%
  \else
    \ifnum#1<20
      十\chinesenumeral{\numexpr#1-10\relax}%
    \else
      \ifnum#1<30
        二十\chinesenumeral{\numexpr#1-20\relax}%
      \else
        \ifnum#1<40
          三十\chinesenumeral{\numexpr#1-30\relax}%
        \else
          \ifnum#1<50
            四十\chinesenumeral{\numexpr#1-40\relax}%
          \else
            \ifnum#1<60
              五十\chinesenumeral{\numexpr#1-50\relax}%
            \else
              \ifnum#1<70
                六十\chinesenumeral{\numexpr#1-60\relax}%
              \else
                \ifnum#1<80
                  七十\chinesenumeral{\numexpr#1-70\relax}%
                \else
                  \ifnum#1<90
                    八十\chinesenumeral{\numexpr#1-80\relax}%
                  \else
                    \ifnum#1<100
                      九十\chinesenumeral{\numexpr#1-90\relax}%
                    \fi
                  \fi
                \fi
              \fi
            \fi
          \fi
        \fi
      \fi
    \fi
  \fi
}


% *-----------------------------------------------------------------------*
% | Table of contents & Chapter management                                |
% *-----------------------------------------------------------------------*

\renewcommand{\figurename}{圗} % So figures would be labeled with 圗 instead of "figure"

% * * * Now for the table of contents
\renewcommand{\contentsname}{目錄} % Traditional Chinese characters for "Contents"
\setcounter{secnumdepth}{0} % no numbering for sections 
% Increase chapter title size in TOC
\usepackage{tocloft} % For customizing table of contents

% Redefine how chapter numbers are displayed in the TOC using Soochow numerals

% comment this out if you don't want to use the soochow numerlas
\renewcommand{\numberline}[1]{\soochowNumeral{#1}\hspace{1em}} % Use Soochow numerals for TOC chapter numbers
\makeatother


% create index - run \makeindex in the document
\usepackage{makeidx}
\makeindex



% % Custom chapter title formatting with Chinese numeral chapter numbers
\usepackage{titlesec}

% Custom chapter title formatting with Chinese numeral chapter numbers
\titleformat{\chapter}[block] % 'block' means the title appears on a new line
  {\Huge\bfseries} % Font size and bold formatting for the title
  {\soochowNumeral{\thechapter}} % Chinese character for chapter number
  {1em} % Space between the number and the title
  {\Huge} % Custom style for the chapter title itself (can modify)



% *-----------------------------------------------------------------------*
% | Document begins                                                       |
% *-----------------------------------------------------------------------*

\title{An Outline of the Siumatgwoon \\ 初論兆物觀}
\author{Hakuryo 白龍}
\date{〢〇〢〥年〢〥月〢〇日}


\begin{document}
% Avoid overfull hbox
\sloppy

% Initialize jyutcitzi processing
\jcz{}

% Generate table of contents

\maketitle

% \subsection*{Question 1:}
% From Rosen, Discrete Mathematics and its Applications: When three professors are seated in a restaurant, the hostess asks them: "Does everyone want coffee?" The first professor says: "I do not know." The second professor then says: "I do not know." Finally, the third professor says: "No, not everyone wants
% coffee." The hostess comes back and gives coffee to the professors who want it. How did she figure out who wanted coffee?
% \subsection*{Answer:}
% The question is : "Does everyone want coffee?" . For the first professor, he wants a coffee because if he doesn't want a coffee, his answer to the question should be "no". Similar to the second professor, he wants a coffee too. For the third professor, he knows that the first two professors want coffee and his answer is "no, not everyone wants coffee", which mean he doesn't want coffee. In conclusion, the first two professors want coffee.
% \subsection*{Question 2:}
% This problem is adapted from Problem 4, page 10 of the Liebeck text.
% Start with the statement A $\rightarrow$ B . Which of the following statements does this statement imply? You should be able to explain your answers using ordinary language.
% \\
% (a) A is true or B is true
% \\
% (b) $\overline{A}$ $\rightarrow$ B
% \\
% (c) B $\rightarrow$ A
% \\
% (d) $\overline{B}$ $\rightarrow$ A
% \\
% (e) $\overline{A}$ $\rightarrow$  $\overline{B}$
% \\
% (f) $\overline{B}$ $\rightarrow$  $\overline{A}$

% \subsection*{Answer:}
% The statement A $\rightarrow$ B means A implies B. 
% \\
% This statement is false if and only if A is true and B is false.
% \\
% Look at these given statements we can see statement (f) is the best choice because (f) is false if and only if negation of B is true and negation of A is false which means B is false and A is true.
% \\
% Therefore, statement (f)  $\overline{B}$ $\rightarrow$  $\overline{A}$ is equivalent to statement A $\rightarrow$ B

% Include chapters
\section{Definitions}

\epigraph{今也,南蠻鴃舌之人,非先王之道。}{——《孟子.滕文公上》}

Consider the old siumatgwoon axioms, and a new relation "synonym", represented by ~, read as "is synonymous with". It satisfies the following axioms.

S1. (reflexivity) for all $x\in S$, $x\sim x$.
S2. (symmetry) For all, $a,b\in S$, if $a\sim b$ then $b\sim a$.
S3. (transitivity) for all $a,b,c \in S$, if $a\sim b$, and $b \sim c$, then $a \sim c$.

S4. (Compositional congruence) If $a \sim a'$, then (if they $ax$ or $xa$ exists):

- $ax \sim a'x$,
- $xa \sim xa'$

S5. Composition cancellation

- If $a * b \sim a * c$, then $b \sim c$;
- If $b∗a\sim c∗a$ then  $b \sim c$;

S6. Divisier compatibility: If $a \sim b$, then for all $x \in S$, $x|a$ iff $x|b$.

S7: Upwards divisibility : If $a∼b$, then for all if $a|x$ then $b|x$.

A Siumatgwoon is Synonym-Closed if in which if $a∗b$  exists and $a∼a'$ and $b∼b'$ then $a'*b, a*b', a'*b'$ also exist. Most siumatgwoons, including the Sinoglyphs, are not synonym-closed. 

Because of S4, Synonym-Closed Siumatgwoons must have $a*b \sim a'*b \sim  a*b'\sim  a'*b'.$ 

% ### **S8: Synonym Replacement and Existence Preservation**

% - 

% Synonym replacement completion: if $a*b$ exists, and if $b\sim b^{’}$, then $a*b^{'}$ exists in S as well.

% Prior to this, we are rather ambiguous and coy as to whether a and b being synonymous means b*x necessarily exists if a*x exists. I feel we should admit closure for synyonym substitutional compositions. We can consider other structure where this axiom is not admitted later.

% S9. Multidecomposition if $ab \sim cd$ and $cd = e$ then $ab =e$. 

% From S4 you'd be able to prove that $a \sim a'$ and $b\sim b'$ then $a * b \sim a' * b'$.

% - The quotient siumatgwoon $S/\sim$
    
%     ## 1. Starting Point: Synonymy ~ as an Equivalence Relation
    
%     Let $S$ be a Siumatgwoon, and let $\sim$ be a synonymy relation on $S$ satisfying:
    
%     - $\sim$ is an equivalence relation: reflexive, symmetric, transitive.
%     - $\sim$ is **compatible with composition**:
        
%         If $a\sim a^{'}$, $b\sim b^{'}$, then $a*b\sim a^{'}*b^{'}$.
        
%     - $\sim$ is **compatible with divisibility**:
        
%         If $a\sim b$, then $x∣a  ⟺  x∣b$ for all  $x\in S$.
        
    
%     These give us a solid foundation to define a quotient structure.
    
%     ---
    
%     ## The set $S/\sim$
    
%     Let:
    
%     $S/∼=\{[a]:a\in S\}$
    
%     where $[a]=\{x∈S∣x\sim a\}$ is the **equivalence class** of $a$.
    
%     ---
    
%     ## 3. Defining the Operations on $S/\sim$
    
%     ### **Multiplication:**
    
%     We define:
    
%     $[a]∗[b]:=[a∗b]$, if $a*b$ exists.
    
%     This is well-defined **because of compositional congruence** (axiom S4′).
    
%     It is also because that we required if $a*b$ exists, so must $a^{'}*b^{'}$.
    
%     That is: if $a \sim a^{'}$, and $b \sim b^{'}$, then:
    
%     $a∗b∼a^{'}*b^{'}\Rightarrow [a∗b]=[a^{'}*b^{'}]$
    
%     So the result is independent of the representative.
    
%     ---
    
%     ### **Divisibility:**
    
%     We define:
    
%     $[a]∣[b] \iff  a∣b$
    
%     Again, this is **well-defined** thanks to the **divisibility equivalence** axiom (S5):
    
%     - If $a\sim a^{'}, b \sim b^{'}$, given we have the axiom that  $a∣b  ⟺  a^{'}|b^{'}$ then:
    
%     $[a]*[b] ⟺  [a^{'}]∣[b^{'}]$
    
%     Thus, | descends to the quotient.
    
%     ---
    
%     ## 4. Verifying Siumatgwoon Axioms in $S/∼$
    
%     Let’s verify that the **quotient structure inherits** the original Siumatgwoon axioms:
    
%     - **Axiom 1 (Reflexivity of** $|$ **)**:
        
%         Since $a∣a⇒[a]∣[a]$
        
%     - **Axiom 2 (Totality)**:
        
%         For all $[a],[b]$, either $[a]∣[b]$ or not.
        
%         This follows since $|$ on $S$ has totality, and divisibility is preserved under $\sim$.
        
%     - **Axiom 3 (Transitivity)**:
        
%         Suppose $[a]∣[b]$ and  $[b]∣[c]$. Recall the definition that $[a]∣[b]$ then $a∣b$. It’s straightforward.
        
%     - **Axiom 4 (Composition and divisability)**:
        
%         We want to prove that if $[a]∗[b]=[c]$, then $[a],[b] | [c]$.  By definition $[a]∗[b]=[c]$ then $a*b\sim c$, and so $a|c$. By the definition of constitutiveness over synonym classes, $[a]|[c]$. The argument same goes for $[b]|[c]$.
        
%         Now come axioms 5 and axioms 6. Do they hold in $S/ \sim$, even if they might not hold in$S$?
        
%         - Axiom 5 (Complete Constructibility and Generators): There exists a non-empty subset $G_{S} \subseteq S$ such that:
%             - (Generation) Every element $x \in S$ is a product of a finite sequence of elements from $G_S$, i.e., $x = g_1 * g_2 * \cdots * g_n,$  where $g_1, \dots, g_n \in G_S$
        
%         And a Simple Siumatgwoon is one where:
        
%         1. $E$ is a generating set, and 
%         2. that decompositions into elementals are unique.
        
%         - Axiom 5 (Complete Constructibility and Generators): There exists a non-empty subset $G_{S} \subseteq S$ such that:
%             - (Generation) Every element $x \in S$ is a product of a finite sequence of elements from $G_S$, i.e., $x = g_1 * g_2 * \cdots * g_n,$  where $g_1, \dots, g_n \in G_S$
%         - Axiom 6. (Finite constitution). For any $x \in S$, there are only finitely many objects $y_1, y_2, \ldots, y_n \in S$ such that  $y_1, y_2, \ldots, y_n | x.$
        
%         Let’s proof that axiom 5 holds first. And then axiom 6 holds. Then we prove that $S/\sim$ is a simple siumatgwoon.
        
%         Consider some element $[x] \in S/\sim$. Given $S$ is a siumatgwoon, we know that any $x$ has a finite decomposition $g_1 * g_2 * g_3 * \cdots g_n$. Given $x=g_1 * g_2 * g_3 * \cdots g_n$, we have $[x]=[g_1 * g_2 * g_3 * \cdots g_n]$, and by the definition of $[*]$ we have $[x]=[g_1 * g_2 * g_3 * \cdots g_n]=[g_1] * [g_2] * [g_3] * \cdots [g_n]$. Given this applies to every $x$ in $S$, it is true of every $[x]\in S/\sim$. 
        
%         Note that some of the $[g_i]$ might be equal to some other $[g_j]$ if $g_i \sim g_j$.
        
%         Now for axiom 6.
        
%         Consider $[x]\in S/\sim$. By axiom 6, there only finitely many $y_1, y_2, \ldots, y_n | x.$ Therefore there are only finitely many $[y_1], [y_2], \ldots, [y_n] | [x].$ Note again, that because there could be $y_i \sim y_j$, then the most reduced list of constituents for $[x]$ might be shorter than $n$.
        
%         Axiom 5 says
        
%     - **Axiom 5 (Generators)**:
        
%         If G⊆SG \subseteq SG⊆S is a generating set, then [G]={[g]∣g∈G}⊆S/∼[G] = \{ [g] \mid g \in G \} \subseteq S/{\sim}[G]={[g]∣g∈G}⊆S/∼ generates all of S/∼S/{\sim}S/∼
        
    
%     Hence, S/∼S/{\sim}S/∼ is a well-defined **quotient Siumatgwoon**.
    
%     ---
    
%     ## 🌀 5. Interpretation of S/∼S/{\sim}S/∼
    
%     - **Elements of S/∼S/{\sim}S/∼** are **semantic classes**: bundles of expressions that mean the same thing.
%     - **Operations and structure** are all inherited from how the parts combine.
%     - This is essentially a move from **syntax to semantics** — a form of canonical simplification.
    
%     ---
    
%     ## 🧠 Bonus: Identity and Atoms in S/∼S/{\sim}S/∼
    
%     - The set of **atomic classes** would be:
    
%     A/∼={[a]∣a∈A}A/{\sim} = \{ [a] \mid a \in A \}
    
%     A/∼={[a]∣a∈A}
    
%     If synonymy is strong (e.g., if all synonyms of atoms are equal), then this is a **true set of atomic meanings**.
    
%     - The quotient set **removes redundancy**: if two characters are written differently but function identically, they’re collapsed.
% - A hierarchy of synonym relations
    
%     Intuitively speaking, the ~ of 言~文 that allows us write 信~伩 seems different from the ~ in 女~巫, as it does not seem the case that 女 can be exchanged for 巫 in all chinese characters. This suggests that there is a hierarchy, or rather, a family of synonymous relations, perhaps context dependent. 
    
%     Some synonymns are more closely clustered together than others. This suggests there is a hierarchy of synonymns.
    
% - A foray into the axioms of the Siumatgwoon - 31st March 2025
    
%     A Siumatgwoon, or a Metaphysic, is a set $S$, paired with the binary relations *composition* $* : S \times S \rightarrow S$ and *constitution* $|:S\times S \rightarrow \{True, False\}$ , such that the following axioms hold. 
    
%     - Axiom 1 (Reflexivity): For all $a \in S$, $a|a$.
%     - Axiom 2 (Totality): For all $a,b\in S$, exactly one of the following holds: $a|b$ or not $a\not|b$
%     - Axiom 3 (Transitivity): For all $a,b,c\in S$, $a|b$ and $b|c$ implies $a|c$.
%     - Axiom 4: If $a*b=c$ for some $a,b,c\in S$, then
%     $a|c$ and $b|c$.
    
%     Some further clarifications: 
    
%     - $a\not|b$ iff $a|b = false$
    
%     Some examples of Siumatgwoons: 
    
%     1. The Chinese characters, **字**
%         1. The Chinese characters, which inspired this whole mathematical exercise, is clearly a Siumatgwoon. If we exercise the synonym exchange of “Siumatgwoon” with “Metaphysic”, this is to say, that the Chinese characters is a Metaphysic. Unfortunately we can’t really *prove* that the Chinese characters are indeed a siumatgwoon, given there’s an infinite number of them, and we do not have a generating rule for all Chinese characters. However, the fact that any subset of Chinese characters is a siumatgwoon in of itself, lends us confidence - perhaps there’s a theorem there waiting to be proved?
%     2. The Roman Numerals $\mathfrak{R}$
        
%         One can clearly see that Roman Numerals $\mathfrak{R}$ are a Siumatgwoon. However, to appreciate the characteristics that make it a Siumatgwoon, let us consider the subset of Roman Numerals from 1 to 10, which we shall show to also be a Siumatgwoon.
        
%         $\mathfrak{R}_{1,10} = \{I, II, III, IV, V, VI, VII, VIII, IX, X\}$
        
%         We will say that for two elements $a,b \in \mathfrak{R}_{1,10}$, $a|b$ iff the glyph $a$ appears in $b$. As such, we can say $I | II$ and $I|III$ as an example, and that $V|IV$ and $X|IX$. 
        
%         For any elements $a,b\in \mathfrak{R}_{1,10}$, if the glyphs $ab$ so written together forms a glyph that also appears in $\mathfrak{R}_{1,10}$, then we’d say that $a*b\in \mathfrak{R}$.
        
%         Now, it’s clear that Ax 1 is satisfied trivially. 
        
%         Ax 2 is also satisfied trivially.
        
%         Ax 3 is also satisfied. 
        
%         Ax 4 is also satisifed. As an example: $I | III, III | VIII$ and we have $I|VIII$.
        
%         So therefore, $\mathfrak{R}_{1,10}$ is a Siumatgwun. 
        
%         It is also interesting to note that as per the definition of $\mathfrak{R}_{1,10}$, it is not compositionally closed. For example, $II * III$ is not in $\mathfrak{R}_{1,10}$. This makes the Siumatgwoon different from a group, where all compositions are contained inside the group. Intuitively, perhaps this suggests the Siumatgwoon is less rich in structure that the mathematical group? Also, note that what $II * III$ should be in $\mathfrak{R}_{1,10}$ is represented by $V$. Intuitively, we can feel that in some sense, $II * III = V$ - that they’re synonymous, identical, referring to the same referent. This is not unlike the presence of variant characters in the Sinoglyphs, such as 體 (body, object)=骵=躰=体, or信 (trust) = 𬢭 = 伩 = 訫 = 㐰… Intuition should hint that this will yield some interesting structures if we pursue the investigation down this path.
        
    
%     1. **(Any?) Numerals system** 
%         1. The fact that the Roman Numerals are a Siumatgwoon should intuitively suggest that any numeral system is a Siumatgwoon. In fact, let us consider the world’s many numeral systems, and see if there is one where it is not a siumatgwoon. 
        
%         |  | 0 | 1 | 2 | 3 | 4 | 5 | 6 | 7 | 8 | 9 |
%         | --- | --- | --- | --- | --- | --- | --- | --- | --- | --- | --- |
%         | 唐字數字 | 〇 | 一 | 二 | 三 | 四 | 五 | 六 | 七 | 八 | 九 |
%         | 唐字數字大寫 | 零 | 壹、弌 | 貳 | 叄 | 肆、[䦉](https://www.unicode.org/cgi-bin/GetUnihanData.pl?codepoint=4989) | 伍 | 陸 | 柒 | 捌 | 玖 |
%         | 字喃 |  | [𠬠](https://www.unicode.org/cgi-bin/GetUnihanData.pl?codepoint=20B20) | [𠄩](https://www.unicode.org/cgi-bin/GetUnihanData.pl?codepoint=20129) | [𠀧](https://www.unicode.org/cgi-bin/GetUnihanData.pl?codepoint=20027) | [𦊚](https://www.unicode.org/cgi-bin/GetUnihanData.pl?codepoint=2629A) | [𠄼](https://www.unicode.org/cgi-bin/GetUnihanData.pl?codepoint=2013C) | [𦒹](https://www.unicode.org/cgi-bin/GetUnihanData.pl?codepoint=264B9) | [𦉱](https://www.unicode.org/cgi-bin/GetUnihanData.pl?codepoint=26271) | [𠔭](https://www.unicode.org/cgi-bin/GetUnihanData.pl?codepoint=2052D) | [𠃩](https://www.unicode.org/cgi-bin/GetUnihanData.pl?codepoint=200E9) |
%         | 蘇州碼子 | 〇 | 〡、一 | 〢、二 | 〣、三 | 〤 | 〥(〡〇) | 〦 | 〧 | 〨 | 〩(〡〤) |
%         | Roman Numerals |  | I | II | III | IV | V | VI | VII | VIII | IX |
%         | Eastern Arabic | ٠ | ١ | ٢ | ٣ | ٤ | ٥ | ٦ | ٧ | ٨ | ٩ |
%         | Persian | ٠ | ۰ | ۱ | ۲ | ۳ | ۴ | ۵ | ۶ | ۷ | ۸ |
%         | Devanagari | ० | १ | २ | ३ | ४ | ५ | ६ | ७ | ८ | ९ |
%         | [Gujarati](https://en.wikipedia.org/wiki/Gujarati_numerals) | ૦ | ૧ | ૨ | ૩ | ૪ | ૫ | ૬ | ૭ | ૮ | ૯ |
%         | [Tibetan](https://en.wikipedia.org/wiki/Tibetan_numerals) | ༠ | ༡ | ༢ | ༣ | ༤ | ༥ | ༦ | ༧ | ༨ | ༩ |
%         | Hebrew |  | א‎ (alef) | ב‎ (bet) | ג‎ (gimel) | ד‎ (dalet) | ה‎ (he) | ו‎ (vav) | ז‎ (zayin) | ח‎ (ḥet) | ט‎ (tet) |
%         | Chinese counting rods |  | 𝍠 | 𝍡 | 𝍢 | 𝍣 | 𝍤 | 𝍥 | 𝍦 | 𝍧 | 𝍨 |
%         | counting 正 |  | 𝍲 | 𝍳 | 𝍴 | 𝍵 | 𝍶 | 𝍶𝍲 | 𝍶𝍳 | 𝍶𝍴 | 𝍶𝍵 |
%         | Tangut  |  | 𘈩 | 𗍫 | 𘕕 | 𗥃 | 𗏁 | 𗤁 | 𗒹 | 𘉋 | 𗢭 |
        
%         I don’t think there’s a single one that’s not a siumatgwoon! Most of them are pathological for sure, in the sense that nothing is constituted by anything else, but none of them violate the Siumatgwoon axioms! 
        
%         The case of the numerals as a Siumatgwoon, or a Metaphysic, is interesting. Numerals all refer to the same referents, the same “things” or “objects”, namely, numbers. However, the glyphs in a given numeral system are themselves imbued with a particular set of metaphysical prejudices and judgements. Under the Roman Numeral Metaphysic, the number 3 is composed of 1 and 2, or composed of three 1s. 4 is composed of 1 and 5, but not 3 and 2. 
        
%     2. The polygons $\mathcal{P}$
        
%         Consider the following graph. If we take all the polygons, convex and star, as elements in a set called $\mathcal{P}$, we can see that it forms a Siumatgwoon.  We state this without formal proof for the infinite set $\mathcal{P}$, but from the subset displayed in the graph below, we can see it is indeed true. A polygon $a$ constitutes polygon $b$ if $a$ appears in $b$$\{3\}$, the equilateral triangle, appears in $\{6/2\}$, the star of David, and so $\{3\} |\{6/2\}$
        
%         The Schläfli symbol is a recursive description, starting with $\{p\}$ for a $p$-sided regular polygon that is convex. For example, $\{3\}$ is an equilateral triangle, $\{4\}$ is a square, $\{5\}$ a convex regular pentagon, etc.
        
%         Regular star polygons are not convex, and their Schläfli symbols take the form $\{p/q\}$, where $p$ is the number of vertices and $q$ is their turning number. Equivalently,  $\{p/q\}$ is created from the vertices of  $\{p\}$ by connecting every  $q$th vertex. For example,  $\{5/2\}$ is a pentagram, while  $\{5\}$ is a pentagon.
        
%         Note that in $p$ and $q$ must be [coprime](https://en.wikipedia.org/wiki/Coprime), or the figure will degenerate, in which case we have the following theorem:
        
%         $\{p/q\}=d\{ \frac{p}{d} / \frac{q}{d}  \}$, where $d=gcd(p,q)$.
        
%         Let us define for any schlaffi symbol $\{p\} | n\{p\}$ for any n. It is intuively true.
        
%         Then clearly axiom 1 is satisfied. Axiom 2 is also satisfied. If 
        
         
        
    
%     ![image.png](attachment:d1cc34e5-6651-44b8-bf5b-1f5cdba73aac:image.png)
    
%     ![image.png](attachment:4d2e52da-b773-4554-88e2-8e9cce7fd6f1:image.png)
    
%     There are 3 flavours of * in Propositional Logics, $\wedge, \vee →$. 
    
%     And we define $\phi | \psi$ if $\phi$ appears in $\psi$, for any wff $\phi, \psi$.
    
%     Then clearly all 4 of the core siumatgun axioms are satisfied. 
    
%     1. Trivial that all $\phi | \phi$
%     2. Also trivial that for any $\phi, \psi$, either $\phi|\psi$ or $\phi\not|\psi$.
%     3. Trivial as well that for any $\phi,\psi | \phi * \psi$.
%     4. Also trivial that if $\phi | \psi$ and $\psi | \theta$ then $\phi | \theta$.
    
%     Propositional Logic as a Siumatgoon has multiple interesting properties: 
    
%     1. It is compositionally complete. 
%     2. It is also constitutionally complete. 
%     3. Is it a simple siumatgun? Are decompositions finite? Yes. Are deccmpositions unique? Yes, up to reordering. So yeah, it’s a simple siumatgwoon.

\end{document}
