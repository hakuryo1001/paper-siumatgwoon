% Import fonts.tex in main.tex only: Include 
% Set CJK main font (for Chinese/Japanese/Korean characters)

% \setmainfont{Times New Roman}
\setCJKmainfont{BabelStone Han}
% doesn't work
% \setCJKmainfont{JyutcitziWithSourceHanSerifTCRegular}[
% Renderer=Basic,
% UprightFont = * ,
% FallbackFonts={BabelStone Han}
% ]



% You can also use \newfontfamily for custom non-CJK fonts if needed
% \setCJKmainfont{JyutcitziWithPMingLiURegular}[Path = ./, Extension = .ttf]
% \setCJKmainfont{JyutcitziWithSourceHanSerifTCRegular}[Path = ./, Extension = .ttf]



\newfontfamily{\jczPMingLiU}{JyutcitziWithPMingLiURegular}[Path = ./fonts/, Extension = .ttf]
% This has the best rendition for latin characters 
\newfontfamily{\jcz}{JyutcitziWithSourceHanSerifTCRegular}[Path = ./fonts/, Extension = .ttf]
\newfontfamily{\batang}{batang}[Path = ./fonts/, Extension = .ttf]
\newCJKfontfamily\koreanfont{Batang}[Path = ./fonts/, Extension = .ttf]
\newfontfamily{\taigi}{GentiumBookPlus-Regular}[Path = ./fonts/, Extension = .ttf]

% \newfontfamily{\biaoyinzi}{Biaoyinzi-2016A}[Path = ./fonts/, Extension = .ttf]

% Load ruby package for furigana (Ruby text)
\usepackage{ruby} or 
% Set CJK main font (for Chinese/Japanese/Korean characters)

% \setmainfont{Times New Roman}
\setCJKmainfont{BabelStone Han}
% doesn't work
% \setCJKmainfont{JyutcitziWithSourceHanSerifTCRegular}[
% Renderer=Basic,
% UprightFont = * ,
% FallbackFonts={BabelStone Han}
% ]



% You can also use \newfontfamily for custom non-CJK fonts if needed
% \setCJKmainfont{JyutcitziWithPMingLiURegular}[Path = ./, Extension = .ttf]
% \setCJKmainfont{JyutcitziWithSourceHanSerifTCRegular}[Path = ./, Extension = .ttf]



\newfontfamily{\jczPMingLiU}{JyutcitziWithPMingLiURegular}[Path = ./fonts/, Extension = .ttf]
% This has the best rendition for latin characters 
\newfontfamily{\jcz}{JyutcitziWithSourceHanSerifTCRegular}[Path = ./fonts/, Extension = .ttf]
\newfontfamily{\batang}{batang}[Path = ./fonts/, Extension = .ttf]
\newCJKfontfamily\koreanfont{Batang}[Path = ./fonts/, Extension = .ttf]
\newfontfamily{\taigi}{GentiumBookPlus-Regular}[Path = ./fonts/, Extension = .ttf]

% \newfontfamily{\biaoyinzi}{Biaoyinzi-2016A}[Path = ./fonts/, Extension = .ttf]

% Load ruby package for furigana (Ruby text)
\usepackage{ruby} in main.tex to make sure \jcz and any font-related settings are loaded there.

% Remove fonts.tex from layout.tex: Since layout.tex will be used as part of main.tex, you don't need to import fonts.tex again within layout.tex. Any definitions from fonts.tex, like \jcz, will be accessible in layout.tex when it's used in main.tex.

% Load layout.tex in main.tex after importing fonts.tex: This ensures that any commands in layout.tex that depend on fonts.tex (like \jcz) are available. Your main.tex should look something like this:


% *-----------------------------------------------------------------------*
% | Formatting                                                             |
% *-----------------------------------------------------------------------*
% Set global paragraph indentation and spacing for standard math papers
\setlength{\parindent}{1em} % Standard paragraph indentation
\setlength{\parskip}{0pt}   % No space between paragraphs

% Footnote styling
\makeatletter
\renewcommand{\@makefntext}[1]{\jcz{\@thefnmark.} #1}
\makeatother

% List spacing control
\setlist[itemize]{topsep=0.5em, itemsep=0.2em, left=1.5em}
\setlist[enumerate]{topsep=0.5em, itemsep=0.2em, left=1.5em}

% *-----------------------------------------------------------------------*
% | Page layout and headers                                                |
% *-----------------------------------------------------------------------*
\pagestyle{fancy}
\fancyhf{} % Clear all header and footer fields

% Define left and right page numbering
% \fancyfoot[LE]{\thepage} % Left side for even pages
% \fancyfoot[RO]{\thepage} % Right side for odd pages

% Ensure the chapter pages (plain style) also have this layout
\makeatletter
\let\ps@plain\ps@fancy
\makeatother




